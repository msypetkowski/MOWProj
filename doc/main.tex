\documentclass[a4paper]{article}

\usepackage[a4paper,  margin=0.4in]{geometry}

\usepackage{graphicx}
\usepackage{float}
\usepackage{multicol}
\usepackage{hyperref}
\usepackage{longtable}



\usepackage[utf8]{inputenc}
\begin{document}


\title{Prediction of student's alcohol consumption with random forests}

\author{Mikołaj Ciesielski, Michał Sypetkowski}
\setlength\columnsep{0.375in}  \newlength\titlebox \setlength\titlebox{2.25in}
\twocolumn
\maketitle

% \tableofcontents
% \newpage

% \begin{multicols}{2}

\section{Data}
\label{data}

Research is done with dataset: \url{https://www.kaggle.com/uciml/student-alcohol-consumption/}.

The data were obtained in a survey of students
math and portuguese language courses in secondary school.
It contains a lot of interesting social,
gender and study information about students.

The dataset contains in total 395 math-course-students samples and 
649 portuguese-language-students samples. Each sample has 33 attributes.

There are several (382) students that belong to both datasets.
These students can be identified by searching for identical attributes
that characterize each student.

For experiments, we merged both tables.
We replaced students grades attributes
(\texttt{G1}, \texttt{G2} and \texttt{G3})
with
\texttt{M\_G1},
\texttt{M\_G2},
\texttt{M\_G3},
\texttt{P\_G1},
\texttt{P\_G2} and
\texttt{P\_G3}.
    Attribute \texttt{paid} (extra paid classes within the course subject) also concerns particular course, so we created attributes
\texttt{M\_paid} and \texttt{P\_paid}.
Since randomForest package (see section \ref{forest}) doesn't support missing variables handling (e.g. surrogate splits),
they are replaced with mean value in case of grades and by mode value in case of paid attribute.
Dalc and Walc attributes are clustered into one binary attribute -- Drink (see section \ref{clust} for details).
Final dataset consists of 662 examples (36 attributes including class).

\subsection{Clustering Dalc and Walc attributes into one binary attribute}
\label{clust}

We aim to build a model that would perform binary classification --
whether student can be considered drinking alcohol regularly or not.
The dataset provides 2 attributes:
\begin{itemize}
    \item Dalc - workday alcohol consumption (numeric: from 1 - very low to 5 - very high)
    \item Walc - weekend alcohol consumption (numeric: from 1 - very low to 5 - very high)
\end{itemize}
2D histogram visualization is shown on figure \ref{fig:hist2D}.

First, we standardize Dalc and Walc attributes.
Using k-means algorithm (k=2) with initial centers in (-1,-1) and (1,1), we got clustering as shown in figure \ref{fig:clust}.
In the end, we classify 45.3\% of the students as drinking with proposed clustering.
The results of this clustering may be considered proper in the context of common sense.


% TODO: think about following
% We maximize consistency within 2 clusters of data using Silhouette coeficient
% (\url{https://en.wikipedia.org/wiki/Silhouette_(clustering)}).
% We use L2 distance.

\begin{figure}[!hbt]
    \centering
    \includegraphics[page=1,width=0.5\textwidth]{../Rplots.pdf}
    \caption[]{2D histogram of Dalc and Walc attributes
    \label{fig:hist2D}
    }
\end{figure}

\begin{figure}[!hbt]
    \centering
    \includegraphics[page=2,width=0.5\textwidth]{../Rplots.pdf}
    \caption[]{Clustering with k-means algorithm. (1: non-drinking, 2: drinking).
                Attributes Dalc and Walc are standardized.
    \label{fig:clust}
    }
\end{figure}

\section{Measuring single attributes entropy}
\label{xent}
First, we check how the class is dependent on each single nominal attribute.

Figure \ref{fig:nominalIG} shows information gain values
of single splits by each nominal attribute (in the context of class Drink described in section \ref{clust}).
As we can see, sex of the students gives significantly bigger information gain
than the other nominal attributes, hence it may be very important attribute
in classification.

% Table \ref{table:nominalIG} shows information gain values for all nominal attributes.
% \input{../nominalIG}

\begin{figure}[!hbt]
    \centering
    \includegraphics[page=5,width=0.5\textwidth]{../Rplots.pdf}
    \caption[]{Information gain values for all nominal attributes
    \label{fig:nominalIG}
    }
\end{figure}

\begin{figure}[!hbt]
    \centering
    \includegraphics[page=6,width=0.5\textwidth]{../Rplots.pdf}
    \caption[]{P-values for all numeric attributes
    \label{fig:pval}
    }
\end{figure}


Figure \ref{fig:pval} shows p-values of t-test significance for numeric attributes.
From this plot, we can decide which attributes are statistically important
in the context of our classification problem.
We can see that portuguese language grades show larger importance than math grades.
This is expected because we have around 2 times more samples from portuguese language
class, and we fill missing values with mean value for each attribute (see section \ref{data}).



\section{Experiments with single decision trees}
\label{expSingle}

Accuracy of a single decision tree in general shouldn't be better than in case of random forest,
so we trained several single decision trees to establish
an upper limit of desired error rate for random forests.

We used rpart\footnote{\url{https://cran.r-project.org/web/packages/rpart/rpart.pdf}} package for the experiments.
We trained many models (with different parameters).
We experimented with:
\begin{enumerate}
    \item cp - complexity parameter.
        Any split that does not decrease the overall lack of fit by a factor of cp is not attempted
        (the lower cp is, the more complex the model is)
    \item minbucket - minimum number of observations in any terminal node
    \item maxdepth - maximum tree depth
    \item split - method for calculating how good the split is(Gini index or entropy)
\end{enumerate}
In accuracy measuring, we use cross-validation with 8 partitions
and 10 repetitions (random shuffles of dataset before partitioning).
% Figure \ref{fig:single1} shows best tree for math studens,
% figure \ref{fig:single2} for portuguese-language studens.
% 
\subsection{Results}
\label{singleConc}

Results for all trained decision trees are shown in table \ref{table:singleResults}.
Standard deviation, min, and max error values don't reveal any models
particularly stable or standing out in any other way.
Best mean error is around 33\%.
We observed that in case of decision trees with lowest mean error, 2 attributes are most important:
\begin{itemize}
    \item goout - going out with friends (numeric: from 1 - very low to 5 - very high).
        Splits with this attribute are done by comparing value of this attribute to 3.5
        (less or equal 3 or higher than 3).
    \item sex - student's sex (binary: 'F' - female or 'M' - male).
        Trees use generalization, such that female students are less likely to drink alcohol.
\end{itemize}
Additionally, students grades attributes and famrel attribute (quality of family relationships)
shown some usefulness.
Tree generated on the whole dataset with parameters producing lowest mean error is shown in figure \ref{fig:single}.

\begin{figure}[!hbt]
    \centering
    % \includegraphics[trim={2cm 2cm 2cm 2cm},clip,page=3,width=0.5\textwidth]{../Rplots.pdf}
    \includegraphics[trim={1cm 1cm 1cm 1cm},clip,page=3,width=0.5\textwidth]{../Rplots.pdf}
    \caption[]{Single decision tree built with parameters that give lowest mean error
    \label{fig:single}
    }
\end{figure}

% \begin{figure}[]
%     \caption[]{Best single decision tree for math students that we achieved}
%     \centering
%     \includegraphics[page=4,width=1.0\textwidth]{../Rplots.pdf}
%     \label{fig:single1}
% \end{figure}
% 
% \begin{figure}[]
%     \caption[]{Best single decision tree for portuguese students that we achieved}
%     \centering
%     \includegraphics[page=6,width=1.0\textwidth]{../Rplots.pdf}
%     \label{fig:single2}
% \end{figure}
% 
% 
% \newpage
\section{Experiments with random forests}
\label{forest}
We used randomForest\footnote{\url{https://cran.r-project.org/web/packages/randomForest/randomForest.pdf}} package for the experiments.
We trained many models (with different parameters).
In accuracy measuring, we use cross-validation with 8 partitions
and 10 repetitions (similarly as in \ref{expSingle}).
We experimented with:
\begin{enumerate}
    \item ntree - number of trees in forest
    \item nodesize - minimum size (number of associated examples) of terminal nodes.
    % \item seed - seed used when building a tree
    %         In case of small random forests, it may be important to experiment with different random seed.
    %         Some implementations may use separate random generator objects for each tree, so that
    %         even with different data (we use cross validation), structure will be similar.
    %         Just in case, we decided to try different seeds set in moment of calling \texttt{randomForest}
    %         function from randomForest package.
    \item mtry - number of variables (attributes) randomly sampled as candidates at each split.
    \item maxnodes - maximum number of terminal nodes that trees in the forest can have.
\end{enumerate}

\subsection{Results}
\label{singleConc}
Results for all tested random forests are shown in table \ref{table:forestResults}.
Standard deviation, min, and max error values don't reveal any random forests, which are
particularly stable or stand out in any other way.
Best mean error rate is around 30\%.

% Random forests doesn't produce better results than single decision trees.
% Moreover, mtry parameter equals total attributes count.
% In results, such random forests have similar properties to single decision trees.
% We found out, that the trees in such forests use mostly splits as
% in case of best single decision trees (see section \ref{singleConc})
Variable importance plot for a random forest built on the whole dataset
with parameters that give lowest mean error,
is shown in figure \ref{fig:importance}.

Confusion matrix for such parameters is shown in table \ref{table:convMx}.
We can see that in practice, our model can filter out non drinking students
quite efficiently.
However, in the case when a student is drinking by ground truth, we cannot
estimate the class efficiently (we have almost equal probabilities).

\begin{table}[!hbt]
    \caption{Confusion matrix (ratio) for random forest built with parameters giving lowest mean error
    \label{table:convMx}
    }
\begin{center}
    \begin{tabular}{| l | l | l |}
    \hline
        & non drinking & drinking \\
    \hline
        predicted non drinking  & 0.47 & 0.22 \\
        predicted drinking  &  0.08 & 0.23 \\
    \hline
    \end{tabular}
\end{center}
\end{table}


\begin{figure}[!hbt]
    \centering
    \includegraphics[trim={0 0 0 2cm},clip,page=4,width=0.5\textwidth]{../Rplots.pdf}
    \caption[]{Variable (attribute) importance plot for random forest built with parameters giving lowest mean error
    \label{fig:importance}
    }
\end{figure}

\subsection{Detailed mtry parameter experiment}
\label{mtryExp}

We experimented with changing maxnodes parameter using other parameters values
that achieved best results (ntree=500, nodesize=10 and maxnodes=30).
Mean error plot is shown in figure \ref{fig:detailedMtry}.

\begin{figure}[!hbt]
    \centering
    % \includegraphics[trim={2cm 2cm 2cm 2cm},clip,page=3,width=0.5\textwidth]{../Rplots.pdf}
    \includegraphics[trim={0cm 0cm 0cm 0cm},clip,page=9,width=0.5\textwidth]{../Rplots.pdf}
    \caption[]{Mean error with standard deviations for various mtry parameter values
    \label{fig:detailedMtry}
    }
\end{figure}

As we can see, best results (error rate around 30\%) are achieved by values 4, 6, and 12.
In our case, 6 is rounded square of attributes count, and that strategy usually
gives best results in classification random forest
(e.g. randomForest package use it for calculating default mtry value for classification).

\subsection{Detailed maxnodes parameter experiment}

Similarly to mtry parameter experiment (section \ref{mtryExp}), we test maxnodes parameter.
Mean error plot is shown in figure \ref{fig:detailedMaxnodes}.

\begin{figure}[!hbt]
    \centering
    % \includegraphics[trim={2cm 2cm 2cm 2cm},clip,page=3,width=0.5\textwidth]{../Rplots.pdf}
    \includegraphics[trim={0cm 0cm 0cm 0cm},clip,page=12,width=0.5\textwidth]{../Rplots.pdf}
    \caption[]{Mean error with standard deviations for various maxnodes parameter values
    \label{fig:detailedMaxnodes}
    }
\end{figure}

As we can see, best results (error rate around 30\%) are achieved by values 15 and 30.
It is hard in this case to explain why these values works better than the others.

\subsection{Detailed nodesize parameter experiment}

Similarly, we test nodesize parameter.
Mean error plot is shown in figure \ref{fig:detailedNodesize}.

\begin{figure}[!hbt]
    \centering
    % \includegraphics[trim={2cm 2cm 2cm 2cm},clip,page=3,width=0.5\textwidth]{../Rplots.pdf}
    \includegraphics[trim={0cm 0cm 0cm 0cm},clip,page=15,width=0.5\textwidth]{../Rplots.pdf}
    \caption[]{Mean error with standard deviations for various nodesize parameter values
    \label{fig:detailedNodesize}
    }
\end{figure}

As we can see, best results (error rate around 30\%) are achieved by value 10.
It is hard in this case to explain why this value works better than the others.


% \begin{figure}[]
%     \caption[]{Variable (attribute) importance plot for math students}
%     \centering
%     \includegraphics[page=5,width=1.0\textwidth]{../Rplots.pdf}
%     \label{fig:imp1}
% \end{figure}
% 
% \begin{figure}[]
%     \caption[]{Variable (attribute) importance plot for portuguese language students}
%     \centering
%     \includegraphics[page=7,width=1.0\textwidth]{../Rplots.pdf}
%     \label{fig:imp2}
% \end{figure}
% 

\section{Feature selection}
\label{featSel}
We performed experiment where we exclude attributes that
seem not significant in our classification task.
We empirically select the attributes which seem to have some importance in the context of
values shown in figures \ref{fig:nominalIG}, \ref{fig:pval}, \ref{fig:importance},
and attributes used by our best single decision tree (see figure \ref{fig:single}).

We select nominal attributes:
sex,
Fjob,
higher,
famsize,
reason; and numeric:
goout,
P\_G1,
P\_G2,
P\_G3,
M\_G1,
M\_G2,
M\_G3,
studytime,
absences,
freetime,
famrel,
health,
age,
M\_G1,
nursery,
M\_paid,
Mjob.
We repeated the experiments with single trees and random forests.

\subsection{Results}

Results of best classifiers of each type (single tree or random forest) remains similar 
(mean error around 33\% for single trees, and 30\% for random forests).
Since the results are similar to the results without feature selection,
we don't include tables with detailed results.
Moreover, we can say that single trees and random forests handle
the unimportant attributes properly (unimportant as defined at the beginning of section \ref{featSel}).



\section{Conclusion}
Random forests achieve results only slightly better (mean error rate around 30\%)
than single decision trees (33\%).

Since random forest is widely used algorithm with many successes, we suspect
that it is hard or impossible to find significantly better
student alcohol consumption classification rule
with our dataset.
High-precision classification in this case, may require more detailed
student profile or larger dataset.
Furthermore, the data were obtained in a survey.
That leads to a suspicion, that there may be a lot of false information in
what the high-school students have written in a probably anonymous survey.

% Random forests doesn't perform well on this dataset (no better than single decision trees).
% Most attributes are noisy in terms of single splits (cross entropy -- see section \ref{xent}).
% Best single decision tree turned out to have only 2 splits (see figure \ref{fig:single}).

% TODO
% Decision trees (and random forests) are trying to use these noisy variables
% by greedily splitting them by some information-gain-like measurement.
% hence their accuracy is similar or lower than accuracy of
% a simple decision tree.


% \end{multicols}

\onecolumn
\newpage
\appendix
\section{All single tree classifiers results}
% \label{singleResults}
% Various models for math studens dataset sorted by error:
\begin{verbatim}
      cp minbucket maxdepth       split     error
4  1e-01         5        5        gini 0.2175628
9  1e-01        10        5        gini 0.2175628
14 1e-01        15        5        gini 0.2175628
19 1e-01         5       15        gini 0.2175628
24 1e-01        10       15        gini 0.2175628
29 1e-01        15       15        gini 0.2175628
44 1e-01        15        5 information 0.2175628
59 1e-01        15       15 information 0.2175628
34 1e-01         5        5 information 0.2267465
39 1e-01        10        5 information 0.2267465
49 1e-01         5       15 information 0.2267465
54 1e-01        10       15 information 0.2267465
13 1e-02        15        5        gini 0.2373430
28 1e-02        15       15        gini 0.2373430
43 1e-02        15        5 information 0.2449372
58 1e-02        15       15 information 0.2449372
33 1e-02         5        5 information 0.2500687
11 1e-04        15        5        gini 0.2510008
12 1e-03        15        5        gini 0.2510008
26 1e-04        15       15        gini 0.2520212
27 1e-03        15       15        gini 0.2520212
38 1e-02        10        5 information 0.2526786
53 1e-02        10       15 information 0.2536990
41 1e-04        15        5 information 0.2540326
42 1e-03        15        5 information 0.2540326
56 1e-04        15       15 information 0.2540326
57 1e-03        15       15 information 0.2540326
...
\end{verbatim}

% 8  1e-02        10        5        gini 0.2577217
% 23 1e-02        10       15        gini 0.2582319
% 36 1e-04        10        5 information 0.2603611
% 37 1e-03        10        5 information 0.2603611
% 3  1e-02         5        5        gini 0.2607535
% 6  1e-04        10        5        gini 0.2638148
% 7  1e-03        10        5        gini 0.2638148
% 31 1e-04         5        5 information 0.2647763
% 32 1e-03         5        5 information 0.2647763
% 51 1e-04        10       15 information 0.2654337
% 52 1e-03        10       15 information 0.2654337
% 21 1e-04        10       15        gini 0.2663658
% 22 1e-03        10       15        gini 0.2663658
% 48 1e-02         5       15 information 0.2739305
% 18 1e-02         5       15        gini 0.2770801
% 1  1e-04         5        5        gini 0.2789443
% 2  1e-03         5        5        gini 0.2789443
% 5  3e-01         5        5        gini 0.3010695
% 10 3e-01        10        5        gini 0.3010695
% 15 3e-01        15        5        gini 0.3010695
% 20 3e-01         5       15        gini 0.3010695
% 25 3e-01        10       15        gini 0.3010695
% 30 3e-01        15       15        gini 0.3010695
% 35 3e-01         5        5 information 0.3010695
% 40 3e-01        10        5 information 0.3010695
% 45 3e-01        15        5 information 0.3010695

\begin{verbatim}
50 3e-01         5       15 information 0.3010695
55 3e-01        10       15 information 0.3010695
60 3e-01        15       15 information 0.3010695
16 1e-04         5       15        gini 0.3054749
17 1e-03         5       15        gini 0.3054749
46 1e-04         5       15 information 0.3119604
47 1e-03         5       15 information 0.3119604
\end{verbatim}


Various models for portugese-language studens dataset sorted by error:
\begin{verbatim}
      cp minbucket maxdepth       split     error
4  1e-01         5        5        gini 0.2234041
9  1e-01        10        5        gini 0.2234041
14 1e-01        15        5        gini 0.2234041
19 1e-01         5       15        gini 0.2234041
24 1e-01        10       15        gini 0.2234041
29 1e-01        15       15        gini 0.2234041
34 1e-01         5        5 information 0.2234041
39 1e-01        10        5 information 0.2234041
44 1e-01        15        5 information 0.2234041
49 1e-01         5       15 information 0.2234041
54 1e-01        10       15 information 0.2234041
59 1e-01        15       15 information 0.2234041
13 1e-02        15        5        gini 0.2252296
28 1e-02        15       15        gini 0.2252296
43 1e-02        15        5 information 0.2255382
58 1e-02        15       15 information 0.2255382
38 1e-02        10        5 information 0.2338528
33 1e-02         5        5 information 0.2341614
8  1e-02        10        5        gini 0.2366343
53 1e-02        10       15 information 0.2372478
...
\end{verbatim}

% 23 1e-02        10       15        gini 0.2394121
% 41 1e-04        15        5 information 0.2446364
% 42 1e-03        15        5 information 0.2446364
% 48 1e-02         5       15 information 0.2446515
% 3  1e-02         5        5        gini 0.2449526
% 11 1e-04        15        5        gini 0.2455623
% 12 1e-03        15        5        gini 0.2455623
% 56 1e-04        15       15 information 0.2464845
% 57 1e-03        15       15 information 0.2464845
% 26 1e-04        15       15        gini 0.2480315
% 27 1e-03        15       15        gini 0.2480315
% 36 1e-04        10        5 information 0.2532633
% 37 1e-03        10        5 information 0.2532633
% 18 1e-02         5       15        gini 0.2544941
% 31 1e-04         5        5 information 0.2575918
% 32 1e-03         5        5 information 0.2575918
% 6  1e-04        10        5        gini 0.2634297
% 7  1e-03        10        5        gini 0.2634297
% 1  1e-04         5        5        gini 0.2742284
% 2  1e-03         5        5        gini 0.2742284
% 21 1e-04        10       15        gini 0.2828440
% 22 1e-03        10       15        gini 0.2828440
% 51 1e-04        10       15 information 0.2853056
% 52 1e-03        10       15 information 0.2853056
% 5  3e-01         5        5        gini 0.2973954
% 10 3e-01        10        5        gini 0.2973954
% 15 3e-01        15        5        gini 0.2973954
% 20 3e-01         5       15        gini 0.2973954

\begin{verbatim}
25 3e-01        10       15        gini 0.2973954
30 3e-01        15       15        gini 0.2973954
35 3e-01         5        5 information 0.2973954
40 3e-01        10        5 information 0.2973954
45 3e-01        15        5 information 0.2973954
50 3e-01         5       15 information 0.2973954
55 3e-01        10       15 information 0.2973954
60 3e-01        15       15 information 0.2973954
46 1e-04         5       15 information 0.3074902
47 1e-03         5       15 information 0.3074902
16 1e-04         5       15        gini 0.3189137
17 1e-03         5       15        gini 0.3189137
\end{verbatim}


Various models for math studens dataset sorted by error:
\begin{verbatim}
      cp minbucket maxdepth       split     error
4  1e-01         5        5        gini 0.2175628
9  1e-01        10        5        gini 0.2175628
14 1e-01        15        5        gini 0.2175628
19 1e-01         5       15        gini 0.2175628
24 1e-01        10       15        gini 0.2175628
29 1e-01        15       15        gini 0.2175628
44 1e-01        15        5 information 0.2175628
59 1e-01        15       15 information 0.2175628
34 1e-01         5        5 information 0.2267465
39 1e-01        10        5 information 0.2267465
49 1e-01         5       15 information 0.2267465
54 1e-01        10       15 information 0.2267465
13 1e-02        15        5        gini 0.2373430
28 1e-02        15       15        gini 0.2373430
43 1e-02        15        5 information 0.2449372
58 1e-02        15       15 information 0.2449372
33 1e-02         5        5 information 0.2500687
11 1e-04        15        5        gini 0.2510008
12 1e-03        15        5        gini 0.2510008
26 1e-04        15       15        gini 0.2520212
27 1e-03        15       15        gini 0.2520212
38 1e-02        10        5 information 0.2526786
53 1e-02        10       15 information 0.2536990
41 1e-04        15        5 information 0.2540326
42 1e-03        15        5 information 0.2540326
56 1e-04        15       15 information 0.2540326
57 1e-03        15       15 information 0.2540326
...
\end{verbatim}

% 8  1e-02        10        5        gini 0.2577217
% 23 1e-02        10       15        gini 0.2582319
% 36 1e-04        10        5 information 0.2603611
% 37 1e-03        10        5 information 0.2603611
% 3  1e-02         5        5        gini 0.2607535
% 6  1e-04        10        5        gini 0.2638148
% 7  1e-03        10        5        gini 0.2638148
% 31 1e-04         5        5 information 0.2647763
% 32 1e-03         5        5 information 0.2647763
% 51 1e-04        10       15 information 0.2654337
% 52 1e-03        10       15 information 0.2654337
% 21 1e-04        10       15        gini 0.2663658
% 22 1e-03        10       15        gini 0.2663658
% 48 1e-02         5       15 information 0.2739305
% 18 1e-02         5       15        gini 0.2770801
% 1  1e-04         5        5        gini 0.2789443
% 2  1e-03         5        5        gini 0.2789443
% 5  3e-01         5        5        gini 0.3010695
% 10 3e-01        10        5        gini 0.3010695
% 15 3e-01        15        5        gini 0.3010695
% 20 3e-01         5       15        gini 0.3010695
% 25 3e-01        10       15        gini 0.3010695
% 30 3e-01        15       15        gini 0.3010695
% 35 3e-01         5        5 information 0.3010695
% 40 3e-01        10        5 information 0.3010695
% 45 3e-01        15        5 information 0.3010695

\begin{verbatim}
50 3e-01         5       15 information 0.3010695
55 3e-01        10       15 information 0.3010695
60 3e-01        15       15 information 0.3010695
16 1e-04         5       15        gini 0.3054749
17 1e-03         5       15        gini 0.3054749
46 1e-04         5       15 information 0.3119604
47 1e-03         5       15 information 0.3119604
\end{verbatim}


Various models for portugese-language studens dataset sorted by error:
\begin{verbatim}
      cp minbucket maxdepth       split     error
4  1e-01         5        5        gini 0.2234041
9  1e-01        10        5        gini 0.2234041
14 1e-01        15        5        gini 0.2234041
19 1e-01         5       15        gini 0.2234041
24 1e-01        10       15        gini 0.2234041
29 1e-01        15       15        gini 0.2234041
34 1e-01         5        5 information 0.2234041
39 1e-01        10        5 information 0.2234041
44 1e-01        15        5 information 0.2234041
49 1e-01         5       15 information 0.2234041
54 1e-01        10       15 information 0.2234041
59 1e-01        15       15 information 0.2234041
13 1e-02        15        5        gini 0.2252296
28 1e-02        15       15        gini 0.2252296
43 1e-02        15        5 information 0.2255382
58 1e-02        15       15 information 0.2255382
38 1e-02        10        5 information 0.2338528
33 1e-02         5        5 information 0.2341614
8  1e-02        10        5        gini 0.2366343
53 1e-02        10       15 information 0.2372478
...
\end{verbatim}

% 23 1e-02        10       15        gini 0.2394121
% 41 1e-04        15        5 information 0.2446364
% 42 1e-03        15        5 information 0.2446364
% 48 1e-02         5       15 information 0.2446515
% 3  1e-02         5        5        gini 0.2449526
% 11 1e-04        15        5        gini 0.2455623
% 12 1e-03        15        5        gini 0.2455623
% 56 1e-04        15       15 information 0.2464845
% 57 1e-03        15       15 information 0.2464845
% 26 1e-04        15       15        gini 0.2480315
% 27 1e-03        15       15        gini 0.2480315
% 36 1e-04        10        5 information 0.2532633
% 37 1e-03        10        5 information 0.2532633
% 18 1e-02         5       15        gini 0.2544941
% 31 1e-04         5        5 information 0.2575918
% 32 1e-03         5        5 information 0.2575918
% 6  1e-04        10        5        gini 0.2634297
% 7  1e-03        10        5        gini 0.2634297
% 1  1e-04         5        5        gini 0.2742284
% 2  1e-03         5        5        gini 0.2742284
% 21 1e-04        10       15        gini 0.2828440
% 22 1e-03        10       15        gini 0.2828440
% 51 1e-04        10       15 information 0.2853056
% 52 1e-03        10       15 information 0.2853056
% 5  3e-01         5        5        gini 0.2973954
% 10 3e-01        10        5        gini 0.2973954
% 15 3e-01        15        5        gini 0.2973954
% 20 3e-01         5       15        gini 0.2973954

\begin{verbatim}
25 3e-01        10       15        gini 0.2973954
30 3e-01        15       15        gini 0.2973954
35 3e-01         5        5 information 0.2973954
40 3e-01        10        5 information 0.2973954
45 3e-01        15        5 information 0.2973954
50 3e-01         5       15 information 0.2973954
55 3e-01        10       15 information 0.2973954
60 3e-01        15       15 information 0.2973954
46 1e-04         5       15 information 0.3074902
47 1e-03         5       15 information 0.3074902
16 1e-04         5       15        gini 0.3189137
17 1e-03         5       15        gini 0.3189137
\end{verbatim}



\newpage
\section{All random forest classifiers results}
% \label{forestResults}
% Results for math students
\begin{verbatim}
    ntree nodesize seed mtry maxnodes     error
136   500       10    2   31        5 0.2200255
147    50        1    4   31        5 0.2205063
151    50       10    4   31        5 0.2205357
160   500       10    5   31        5 0.2210165
131    50        1    2   31        5 0.2214678
148   500        1    4   31        5 0.2214973
132   500        1    2   31        5 0.2215267
124   500        1    1   31        5 0.2215856
128   500       10    1   31        5 0.2215856
140   500        1    3   31        5 0.2215856
152   500       10    4   31        5 0.2215856
135    50       10    2   31        5 0.2219780
143    50       10    3   31        5 0.2220663
144   500       10    3   31        5 0.2220663
159    50       10    5   31        5 0.2220663
127    50       10    1   31        5 0.2225765
156   500        1    5   31        5 0.2234498
139    50        1    3   31        5 0.2235086
155    50        1    5   31        5 0.2235381
123    50        1    1   31        5 0.2259125
130     5        1    2   31        5 0.2260302
134     5       10    2   31        5 0.2270506
158     5       10    5   31        5 0.2275903
154     5        1    5   31        5 0.2291209
145     1        1    4   31        5 0.2292975
149     1       10    4   31        5 0.2292975
247    50       10    1   10       10 0.2296311
291    50        1    2   31       10 0.2296900
315    50        1    5   31       10 0.2303179
...
\end{verbatim}

% 146     5        1    4   31        5 0.2310440
% 150     5       10    4   31        5 0.2315542
% 259    50        1    3   10       10 0.2315542
% 316   500        1    5   31       10 0.2317896
% 271    50       10    4   10       10 0.2322115
% 287    50       10    1   31       10 0.2326923
% 284   500        1    1   31       10 0.2327806
% 279    50       10    5   10       10 0.2330553
% 267    50        1    4   10       10 0.2331731
% 292   500        1    2   31       10 0.2332614
% 319    50       10    5   31       10 0.2332614
% 264   500       10    3   10       10 0.2332614
% 427    50        1    4   10       30 0.2337422
% 300   500        1    3   31       10 0.2338010
% 280   500       10    5   10       10 0.2341935
% 244   500        1    1   10       10 0.2342524
% 122     5        1    1   31        5 0.2343112
% 126     5       10    1   31        5 0.2343112
% 307    50        1    4   31       10 0.2343112
% 295    50       10    2   31       10 0.2346743
% 320   500       10    5   31       10 0.2348214
% 303    50       10    3   31       10 0.2348509
% 256   500       10    2   10       10 0.2351845
% 252   500        1    2   10       10 0.2352139
% 263    50       10    3   10       10 0.2352433
% 283    50        1    1   31       10 0.2352433
% 296   500       10    2   31       10 0.2353611
% 248   500       10    1   10       10 0.2357241
% 299    50        1    3   31       10 0.2358418
% 272   500       10    4   10       10 0.2362637
% 436   500        1    5   10       30 0.2363815
% 412   500        1    2   10       30 0.2367445
% 268   500        1    4   10       10 0.2367445
% 428   500        1    4   10       30 0.2367739
% 288   500       10    1   31       10 0.2368034
% 304   500       10    3   31       10 0.2368034
% 308   500        1    4   31       10 0.2368034
% 404   500        1    1   10       30 0.2368034
% 384   500       10    3    6       30 0.2371958
% 260   500        1    3   10       10 0.2371958
% 255    50       10    2   10       10 0.2372253
% 375    50       10    2    6       30 0.2373724
% 251    50        1    2   10       10 0.2376177
% 367    50       10    1    6       30 0.2376766
% 276   500        1    5   10       10 0.2377355
% 420   500        1    3   10       30 0.2378238
% 392   500       10    4    6       30 0.2381279
% 411    50        1    2   10       30 0.2381279
% 408   500       10    1   10       30 0.2382457
% 311    50       10    4   31       10 0.2383340
% 443    50        1    1   31       30 0.2383340
% 275    50        1    5   10       10 0.2386381
% 431    50       10    4   10       30 0.2387559
% 432   500       10    4   10       30 0.2388442
% 312   500       10    4   31       10 0.2388736
% 439    50       10    5   10       30 0.2392072
% 452   500        1    2   31       30 0.2393250
% 403    50        1    1   10       30 0.2394427
% 479    50       10    5   31       30 0.2396291
% 364   500        1    1    6       30 0.2396586
% 400   500       10    5    6       30 0.2396586
% 371    50        1    2    6       30 0.2397174
% 419    50        1    3   10       30 0.2397469
% 399    50       10    5    6       30 0.2397763
% 440   500       10    5   10       30 0.2398352
% 416   500       10    2   10       30 0.2398646
% 391    50       10    4    6       30 0.2398940
% 138     5        1    3   31        5 0.2400805
% 444   500        1    1   31       30 0.2403159
% 243    50        1    1   10       10 0.2408261
% 456   500       10    2   31       30 0.2408850
% 376   500       10    2    6       30 0.2411009
% 314     5        1    5   31       10 0.2411892
% 451    50        1    2   31       30 0.2413658
% 472   500       10    4   31       30 0.2413952
% 306     5        1    4   31       10 0.2414246
% 388   500        1    4    6       30 0.2416994
% 290     5        1    2   31       10 0.2417288
% 423    50       10    3   10       30 0.2418171
% 435    50        1    5   10       30 0.2418465
% 383    50       10    3    6       30 0.2419054
% 142     5       10    3   31        5 0.2421213
% 407    50       10    1   10       30 0.2422684
% 448   500       10    1   31       30 0.2424451
% 447    50       10    1   31       30 0.2425039
% 424   500       10    3   10       30 0.2427786
% 476   500        1    5   31       30 0.2428375
% 480   500       10    5   31       30 0.2428964
% 468   500        1    4   31       30 0.2433477
% 310     5       10    4   31       10 0.2434360
% 387    50        1    4    6       30 0.2437402
% 368   500       10    1    6       30 0.2437696
% 396   500        1    5    6       30 0.2437696
% 475    50        1    5   31       30 0.2437696
% 380   500        1    3    6       30 0.2442210
% 318     5       10    5   31       10 0.2442504
% 372   500        1    2    6       30 0.2447312
% 286     5       10    1   31       10 0.2448489
% 460   500        1    3   31       30 0.2448783
% 282     5        1    1   31       10 0.2454474
% 459    50        1    3   31       30 0.2458104
% 363    50        1    1    6       30 0.2458987
% 455    50       10    2   31       30 0.2459282
% 379    50        1    3    6       30 0.2462029
% 294     5       10    2   31       10 0.2463206
% 298     5        1    3   31       10 0.2463795
% 129     1        1    2   31        5 0.2465267
% 133     1       10    2   31        5 0.2465267
% 471    50       10    4   31       30 0.2469486
% 464   500       10    3   31       30 0.2473705
% 395    50        1    5    6       30 0.2477630
% 467    50        1    4   31       30 0.2479396
% 223    50       10    3    6       10 0.2482143
% 216   500       10    2    6       10 0.2482437
% 418     5        1    3   10       30 0.2484498
% 302     5       10    3   31       10 0.2484792
% 231    50       10    4    6       10 0.2497449
% 240   500       10    5    6       10 0.2502551
% 211    50        1    2    6       10 0.2507064
% 463    50       10    3   31       30 0.2510302
% 235    50        1    5    6       10 0.2511578
% 120   500       10    5   10        5 0.2513049
% 270     5       10    4   10       10 0.2516582
% 232   500       10    4    6       10 0.2517857
% 415    50       10    2   10       30 0.2520801
% 219    50        1    3    6       10 0.2521782
% 115    50        1    5   10        5 0.2522370
% 99     50        1    3   10        5 0.2522959
% 239    50       10    5    6       10 0.2522959
% 218     5        1    3    6       10 0.2523842
% 96    500       10    2   10        5 0.2527767
% 224   500       10    3    6       10 0.2528061
% 119    50       10    5   10        5 0.2528650
% 87     50       10    1   10        5 0.2532575
% 208   500       10    1    6       10 0.2532575
% 88    500       10    1   10        5 0.2532869
% 204   500        1    1    6       10 0.2532869
% 104   500       10    3   10        5 0.2533163
% 95     50       10    2   10        5 0.2537088
% 103    50       10    3   10        5 0.2537088
% 207    50       10    1    6       10 0.2537677
% 236   500        1    5    6       10 0.2537971
% 242     5        1    1   10       10 0.2538265
% 91     50        1    2   10        5 0.2538560
% 250     5        1    2   10       10 0.2541503
% 111    50       10    4   10        5 0.2542190
% 83     50        1    1   10        5 0.2546998
% 212   500        1    2    6       10 0.2547881
% 203    50        1    1    6       10 0.2548764
% 100   500        1    3   10        5 0.2553571
% 246     5       10    1   10       10 0.2554454
% 84    500        1    1   10        5 0.2557496
% 90      5        1    2   10        5 0.2561911
% 116   500        1    5   10        5 0.2562304
% 227    50        1    4    6       10 0.2562892
% 112   500       10    4   10        5 0.2563187
% 215    50       10    2    6       10 0.2563481
% 220   500        1    3    6       10 0.2563481
% 82      5        1    1   10        5 0.2564659
% 474     5        1    5   31       30 0.2566425
% 228   500        1    4    6       10 0.2573685
% 234     5        1    5    6       10 0.2575451
% 108   500        1    4   10        5 0.2578787
% 222     5       10    3    6       10 0.2578787
% 206     5       10    1    6       10 0.2582614
% 442     5        1    1   31       30 0.2586538
% 258     5        1    3   10       10 0.2590757
% 202     5        1    1    6       10 0.2592818
% 254     5       10    2   10       10 0.2594976
% 305     1        1    4   31       10 0.2594976
% 94      5       10    2   10        5 0.2597037
% 92    500        1    2   10        5 0.2598607
% 466     5        1    4   31       30 0.2602433
% 86      5       10    1   10        5 0.2605181
% 114     5        1    5   10        5 0.2605769
% 434     5        1    5   10       30 0.2609105
% 266     5        1    4   10       10 0.2611754
% 262     5       10    3   10       10 0.2613913
% 402     5        1    1   10       30 0.2614403
% 309     1       10    4   31       10 0.2615385
% 121     1        1    1   31        5 0.2624706
% 107    50        1    4   10        5 0.2625294
% 118     5       10    5   10        5 0.2625294
% 125     1       10    1   31        5 0.2629808
% 278     5       10    5   10       10 0.2629808
% 446     5       10    1   31       30 0.2643838
% 458     5        1    3   31       30 0.2656201
% 289     1        1    2   31       10 0.2659537
% 293     1       10    2   31       10 0.2660126
% 410     5        1    2   10       30 0.2675726
% 426     5        1    4   10       30 0.2676903
% 274     5        1    5   10       10 0.2677492
% 153     1        1    5   31        5 0.2680534
% 106     5        1    4   10        5 0.2682594
% 214     5       10    2    6       10 0.2682594
% 374     5       10    2    6       30 0.2683477
% 157     1       10    5   31        5 0.2685636
% 210     5        1    2    6       10 0.2686519
% 226     5        1    4    6       10 0.2690443
% 42      5        1    1    6        5 0.2700648
% 110     5       10    4   10        5 0.2712912
% 378     5        1    3    6       30 0.2714973
% 398     5       10    5    6       30 0.2715365
% 46      5       10    1    6        5 0.2716248
% 462     5       10    3   31       30 0.2734203
% 430     5       10    4   10       30 0.2744702
% 478     5       10    5   31       30 0.2748038
% 406     5       10    1   10       30 0.2752845
% 370     5        1    2    6       30 0.2758536
% 43     50        1    1    6        5 0.2760989
% 414     5       10    2   10       30 0.2762755
% 285     1       10    1   31       10 0.2763344
% 386     5        1    4    6       30 0.2763638
% 47     50       10    1    6        5 0.2764914
% 78      5       10    5    6        5 0.2765502
% 394     5        1    5    6       30 0.2768151
% 55     50       10    2    6        5 0.2770310
% 54      5       10    2    6        5 0.2771193
% 102     5       10    3   10        5 0.2772959
% 51     50        1    2    6        5 0.2780220
% 245     1       10    1   10       10 0.2783163
% 79     50       10    5    6        5 0.2785027
% 450     5        1    2   31       30 0.2787088
% 382     5       10    3    6       30 0.2788854
% 59     50        1    3    6        5 0.2789835
% 454     5       10    2   31       30 0.2791013
% 71     50       10    4    6        5 0.2796409
% 74      5        1    5    6        5 0.2801805
% 281     1        1    1   31       10 0.2803571
% 313     1        1    5   31       10 0.2806613
% 50      5        1    2    6        5 0.2806907
% 48    500       10    1    6        5 0.2816228
% 45      1       10    1    6        5 0.2816523
% 98      5        1    3   10        5 0.2822508
% 233     1        1    5    6       10 0.2828493
% 64    500       10    3    6        5 0.2831240
% 230     5       10    4    6       10 0.2833301
% 80    500       10    5    6        5 0.2836342
% 52    500        1    2    6        5 0.2836342
% 362     5        1    1    6       30 0.2839874
% 56    500       10    2    6        5 0.2841444
% 41      1        1    1    6        5 0.2842622
% 390     5       10    4    6       30 0.2845271
% 67     50        1    4    6        5 0.2851648
% 238     5       10    5    6       10 0.2852531
% 342     5       10    3    1       30 0.2852531
% 241     1        1    1   10       10 0.2854297
% 470     5       10    4   31       30 0.2857633
% 63     50       10    3    6        5 0.2866366
% 44    500        1    1    6        5 0.2866954
% 72    500       10    4    6        5 0.2866954
% 317     1       10    5   31       10 0.2867838
% 62      5       10    3    6        5 0.2872645
% 70      5       10    4    6        5 0.2874706
% 438     5       10    5   10       30 0.2874706
% 76    500        1    5    6        5 0.2876864
% 75     50        1    5    6        5 0.2882261
% 422     5       10    3   10       30 0.2882555
% 330     5        1    2    1       30 0.2882849
% 66      5        1    4    6        5 0.2884615
% 166     5       10    1    1       10 0.2887363
% 58      5        1    3    6        5 0.2887951
% 60    500        1    3    6        5 0.2891876
% 297     1        1    3   31       10 0.2902963
% 137     1        1    3   31        5 0.2904435
% 343    50       10    3    1       30 0.2907182
% 68    500        1    4    6        5 0.2907476
% 366     5       10    1    6       30 0.2907771
% 359    50       10    5    1       30 0.2912578
% 237     1       10    5    6       10 0.2915816
% 186     5        1    4    1       10 0.2917975
% 162     5        1    1    1       10 0.2937206
% 327    50       10    1    1       30 0.2937500
% 141     1       10    3   31        5 0.2945251
% 355    50        1    5    1       30 0.2947410
% 351    50       10    4    1       30 0.2947704
% 323    50        1    1    1       30 0.2947998
% 331    50        1    2    1       30 0.2947998
% 81      1        1    1   10        5 0.2950353
% 85      1       10    1   10        5 0.2950353
% 174     5       10    2    1       10 0.2953984
% 441     1        1    1   31       30 0.2955455
% 170     5        1    2    1       10 0.2957614
% 34      5        1    5    1        5 0.2957614
% 178     5        1    3    1       10 0.2957908
% 335    50       10    2    1       30 0.2963010
% 350     5       10    4    1       30 0.2963010
% 326     5       10    1    1       30 0.2963893
% 334     5       10    2    1       30 0.2964776
% 322     5        1    1    1       30 0.2968112
% 249     1        1    2   10       10 0.2968603
% 347    50        1    4    1       30 0.2972920
% 38      5       10    5    1        5 0.2973214
% 344   500       10    3    1       30 0.2973214
% 360   500       10    5    1       30 0.2973214
% 269     1       10    4   10       10 0.2974294
% 198     5       10    5    1       10 0.2977433
% 358     5       10    5    1       30 0.2977728
% 339    50        1    3    1       30 0.2978022
% 328   500       10    1    1       30 0.2978316
% 336   500       10    2    1       30 0.2978316
% 352   500       10    4    1       30 0.2978316
% 93      1       10    2   10        5 0.2981849
% 348   500        1    4    1       30 0.2983418
% 445     1       10    1   31       30 0.2987834
% 26      5        1    4    1        5 0.2988226
% 301     1       10    3   31       10 0.2988520
% 265     1        1    4   10       10 0.2989600
% 2       5        1    1    1        5 0.2993328
% 324   500        1    1    1       30 0.2993328
% 332   500        1    2    1       30 0.2993328
% 356   500        1    5    1       30 0.2993328
% 14      5       10    2    1        5 0.2993328
% 18      5        1    3    1        5 0.2993328
% 22      5       10    3    1        5 0.2993622
% 10      5        1    2    1        5 0.2993917
% 253     1       10    2   10       10 0.2994996
% 6       5       10    1    1        5 0.2998430
% 190     5       10    4    1       10 0.2998430
% 340   500        1    3    1       30 0.2998430
% 89      1        1    2   10        5 0.3001079
% 182     5       10    3    1       10 0.3003238
% 175    50       10    2    1       10 0.3003532
% 171    50        1    2    1       10 0.3003532
% 191    50       10    4    1       10 0.3003532
% 30      5       10    4    1        5 0.3008634
% 3      50        1    1    1        5 0.3008634
% 4     500        1    1    1        5 0.3008634
% 7      50       10    1    1        5 0.3008634
% 8     500       10    1    1        5 0.3008634
% 11     50        1    2    1        5 0.3008634
% 12    500        1    2    1        5 0.3008634
% 15     50       10    2    1        5 0.3008634
% 16    500       10    2    1        5 0.3008634
% 19     50        1    3    1        5 0.3008634
% 20    500        1    3    1        5 0.3008634
% 23     50       10    3    1        5 0.3008634
% 24    500       10    3    1        5 0.3008634
% 27     50        1    4    1        5 0.3008634
% 28    500        1    4    1        5 0.3008634
% 31     50       10    4    1        5 0.3008634
% 32    500       10    4    1        5 0.3008634
% 35     50        1    5    1        5 0.3008634
% 36    500        1    5    1        5 0.3008634
% 39     50       10    5    1        5 0.3008634
% 40    500       10    5    1        5 0.3008634
% 163    50        1    1    1       10 0.3008634
% 164   500        1    1    1       10 0.3008634
% 167    50       10    1    1       10 0.3008634
% 168   500       10    1    1       10 0.3008634
% 172   500        1    2    1       10 0.3008634
% 176   500       10    2    1       10 0.3008634
% 179    50        1    3    1       10 0.3008634
% 180   500        1    3    1       10 0.3008634
% 183    50       10    3    1       10 0.3008634
% 184   500       10    3    1       10 0.3008634
% 187    50        1    4    1       10 0.3008634
% 188   500        1    4    1       10 0.3008634
% 192   500       10    4    1       10 0.3008634
% 195    50        1    5    1       10 0.3008634
% 196   500        1    5    1       10 0.3008634
% 199    50       10    5    1       10 0.3008634
% 200   500       10    5    1       10 0.3008634
% 205     1       10    1    6       10 0.3025412
% 338     5        1    3    1       30 0.3031397
% 105     1        1    4   10        5 0.3032869
% 257     1        1    3   10       10 0.3053964
% 354     5        1    5    1       30 0.3054258
% 346     5        1    4    1       30 0.3055141
% 194     5        1    5    1       10 0.3060538
% 17      1        1    3    1        5 0.3065051
% 109     1       10    4   10        5 0.3068583
% 201     1        1    1    6       10 0.3071919
% 9       1        1    2    1        5 0.3079768
% 21      1       10    3    1        5 0.3080357
% 97      1        1    3   10        5 0.3084870
% 173     1       10    2    1       10 0.3085165
% 73      1        1    5    6        5 0.3087225
% 77      1       10    5    6        5 0.3087225
% 197     1       10    5    1       10 0.3089384
% 13      1       10    2    1        5 0.3089973
% 5       1       10    1    1        5 0.3090267
% 169     1        1    2    1       10 0.3097135
% 61      1       10    3    6        5 0.3101648
% 217     1        1    3    6       10 0.3103414
% 1       1        1    1    1        5 0.3105573
% 57      1        1    3    6        5 0.3106456
% 161     1        1    1    1       10 0.3120879
% 181     1       10    3    1       10 0.3121173
% 473     1        1    5   31       30 0.3122351
% 465     1        1    4   31       30 0.3132261
% 101     1       10    3   10        5 0.3135891
% 165     1       10    1    1       10 0.3136480
% 273     1        1    5   10       10 0.3138246
% 341     1       10    3    1       30 0.3139521
% 113     1        1    5   10        5 0.3143053
% 117     1       10    5   10        5 0.3143053
% 53      1       10    2    6        5 0.3146978
% 49      1        1    2    6        5 0.3152080
% 177     1        1    3    1       10 0.3156005
% 405     1       10    1   10       30 0.3160714
% 209     1        1    2    6       10 0.3161303
% 377     1        1    3    6       30 0.3170035
% 261     1       10    3   10       10 0.3171605
% 417     1        1    3   10       30 0.3174549
% 221     1       10    3    6       10 0.3177590
% 369     1        1    2    6       30 0.3182005
% 33      1        1    5    1        5 0.3186323
% 37      1       10    5    1        5 0.3186323
% 185     1        1    4    1       10 0.3186911
% 477     1       10    5   31       30 0.3194074
% 401     1        1    1   10       30 0.3195546
% 225     1        1    4    6       10 0.3204278
% 213     1       10    2    6       10 0.3207810
% 333     1       10    2    1       30 0.3213893
% 277     1       10    5   10       10 0.3214482
% 425     1        1    4   10       30 0.3215659
% 353     1        1    5    1       30 0.3216641
% 409     1        1    2   10       30 0.3221350
% 25      1        1    4    1        5 0.3232830
% 29      1       10    4    1        5 0.3232830
% 469     1       10    4   31       30 0.3234007
% 413     1       10    2   10       30 0.3238815
% 357     1       10    5    1       30 0.3241268
% 449     1        1    2   31       30 0.3248724
% 433     1        1    5   10       30 0.3249608
% 229     1       10    4    6       10 0.3255298
% 193     1        1    5    1       10 0.3256279
% 189     1       10    4    1       10 0.3268544
% 457     1        1    3   31       30 0.3270310
% 345     1        1    4    1       30 0.3270703
% 321     1        1    1    1       30 0.3270997
% 361     1        1    1    6       30 0.3303277
% 453     1       10    2   31       30 0.3309360
% 65      1        1    4    6        5 0.3316228
% 69      1       10    4    6        5 0.3326432
% 325     1       10    1    1       30 0.3327414

\begin{verbatim}
393     1        1    5    6       30 0.3341739
397     1       10    5    6       30 0.3343897
389     1       10    4    6       30 0.3344192
385     1        1    4    6       30 0.3348999
337     1        1    3    1       30 0.3353218
329     1        1    2    1       30 0.3372645
461     1       10    3   31       30 0.3381083
365     1       10    1    6       30 0.3445644
421     1       10    3   10       30 0.3486754
429     1       10    4   10       30 0.3498430
381     1       10    3    6       30 0.3517955
437     1       10    5   10       30 0.3529042
373     1       10    2    6       30 0.3552394
349     1       10    4    1       30 0.3581044
\end{verbatim}


Results for portugese language students:
\begin{verbatim}
    ntree nodesize seed mtry maxnodes     error
124   500        1    1   31        5 0.2234380
128   500       10    1   31        5 0.2234380
132   500        1    2   31        5 0.2234380
140   500        1    3   31        5 0.2234380
144   500       10    3   31        5 0.2234380
147    50        1    4   31        5 0.2234380
148   500        1    4   31        5 0.2234380
152   500       10    4   31        5 0.2234380
156   500        1    5   31        5 0.2234380
160   500       10    5   31        5 0.2234380
131    50        1    2   31        5 0.2237466
135    50       10    2   31        5 0.2237466
136   500       10    2   31        5 0.2237466
151    50       10    4   31        5 0.2237466
155    50        1    5   31        5 0.2237466
159    50       10    5   31        5 0.2237466
130     5        1    2   31        5 0.2240477
127    50       10    1   31        5 0.2240553
154     5        1    5   31        5 0.2240553
158     5       10    5   31        5 0.2240553
143    50       10    3   31        5 0.2243601
123    50        1    1   31        5 0.2243639
134     5       10    2   31        5 0.2246575
139    50        1    3   31        5 0.2246688
432   500       10    4   10       30 0.2252748
404   500        1    1   10       30 0.2255646
420   500        1    3   10       30 0.2255759
412   500        1    2   10       30 0.2258732
440   500       10    5   10       30 0.2258770
436   500        1    5   10       30 0.2258808
424   500       10    3   10       30 0.2261932
146     5        1    4   31        5 0.2268255
150     5       10    4   31        5 0.2268255
252   500        1    2   10       10 0.2271304
...
\end{verbatim}

% 260   500        1    3   10       10 0.2271304
% 268   500        1    4   10       10 0.2274390
% 428   500        1    4   10       30 0.2277326
% 264   500       10    3   10       10 0.2277477
% 291    50        1    2   31       10 0.2277552
% 248   500       10    1   10       10 0.2280601
% 280   500       10    5   10       10 0.2280601
% 408   500       10    1   10       30 0.2283461
% 431    50       10    4   10       30 0.2283499
% 272   500       10    4   10       10 0.2283650
% 416   500       10    2   10       30 0.2289559
% 276   500        1    5   10       10 0.2289785
% 122     5        1    1   31        5 0.2289860
% 126     5       10    1   31        5 0.2289898
% 251    50        1    2   10       10 0.2289898
% 244   500        1    1   10       10 0.2292871
% 435    50        1    5   10       30 0.2295845
% 138     5        1    3   31        5 0.2295882
% 142     5       10    3   31        5 0.2295882
% 419    50        1    3   10       30 0.2295958
% 256   500       10    2   10       10 0.2295958
% 403    50        1    1   10       30 0.2298856
% 415    50       10    2   10       30 0.2298856
% 271    50       10    4   10       10 0.2299044
% 423    50       10    3   10       30 0.2301942
% 307    50        1    4   31       10 0.2302243
% 295    50       10    2   31       10 0.2302281
% 384   500       10    3    6       30 0.2305104
% 292   500        1    2   31       10 0.2305367
% 467    50        1    4   31       30 0.2308266
% 375    50       10    2    6       30 0.2311201
% 407    50       10    1   10       30 0.2311277
% 314     5        1    5   31       10 0.2311390
% 308   500        1    4   31       10 0.2311540
% 263    50       10    3   10       10 0.2314363
% 267    50        1    4   10       10 0.2314514
% 283    50        1    1   31       10 0.2314551
% 311    50       10    4   31       10 0.2314589
% 439    50       10    5   10       30 0.2317337
% 255    50       10    2   10       10 0.2317638
% 304   500       10    3   31       10 0.2317675
% 296   500       10    2   31       10 0.2317713
% 379    50        1    3    6       30 0.2320536
% 275    50        1    5   10       10 0.2320574
% 279    50       10    5   10       10 0.2320574
% 303    50       10    3   31       10 0.2320724
% 312   500       10    4   31       10 0.2320762
% 316   500        1    5   31       10 0.2320799
% 259    50        1    3   10       10 0.2323622
% 298     5        1    3   31       10 0.2323622
% 243    50        1    1   10       10 0.2323660
% 319    50       10    5   31       10 0.2323773
% 287    50       10    1   31       10 0.2323773
% 315    50        1    5   31       10 0.2323848
% 320   500       10    5   31       10 0.2323886
% 427    50        1    4   10       30 0.2326558
% 411    50        1    2   10       30 0.2326671
% 247    50       10    1   10       10 0.2326746
% 300   500        1    3   31       10 0.2326972
% 288   500       10    1   31       10 0.2326972
% 376   500       10    2    6       30 0.2329569
% 88    500       10    1   10        5 0.2329908
% 92    500        1    2   10        5 0.2329908
% 100   500        1    3   10        5 0.2329946
% 400   500       10    5    6       30 0.2332731
% 372   500        1    2    6       30 0.2332919
% 104   500       10    3   10        5 0.2332957
% 310     5       10    4   31       10 0.2332957
% 475    50        1    5   31       30 0.2333107
% 318     5       10    5   31       10 0.2333107
% 284   500        1    1   31       10 0.2333145
% 299    50        1    3   31       10 0.2336006
% 451    50        1    2   31       30 0.2336081
% 476   500        1    5   31       30 0.2336119
% 388   500        1    4    6       30 0.2338979
% 380   500        1    3    6       30 0.2338979
% 392   500       10    4    6       30 0.2339017
% 396   500        1    5    6       30 0.2339055
% 116   500        1    5   10        5 0.2339167
% 306     5        1    4   31       10 0.2342103
% 460   500        1    3   31       30 0.2342216
% 84    500        1    1   10        5 0.2342254
% 96    500       10    2   10        5 0.2342291
% 395    50        1    5    6       30 0.2345303
% 112   500       10    4   10        5 0.2345340
% 452   500        1    2   31       30 0.2345340
% 383    50       10    3    6       30 0.2348389
% 459    50        1    3   31       30 0.2348389
% 108   500        1    4   10        5 0.2348427
% 444   500        1    1   31       30 0.2348427
% 120   500       10    5   10        5 0.2348464
% 468   500        1    4   31       30 0.2351588
% 368   500       10    1    6       30 0.2357498
% 364   500        1    1    6       30 0.2357498
% 443    50        1    1   31       30 0.2357573
% 399    50       10    5    6       30 0.2363633
% 391    50       10    4    6       30 0.2366795
% 302     5       10    3   31       10 0.2369919
% 103    50       10    3   10        5 0.2369956
% 456   500       10    2   31       30 0.2372930
% 83     50        1    1   10        5 0.2373005
% 371    50        1    2    6       30 0.2373193
% 480   500       10    5   31       30 0.2376092
% 367    50       10    1    6       30 0.2379065
% 111    50       10    4   10        5 0.2379178
% 107    50        1    4   10        5 0.2379291
% 448   500       10    1   31       30 0.2382340
% 87     50       10    1   10        5 0.2391561
% 387    50        1    4    6       30 0.2394572
% 294     5       10    2   31       10 0.2394723
% 290     5        1    2   31       10 0.2397696
% 479    50       10    5   31       30 0.2397696
% 91     50        1    2   10        5 0.2397734
% 363    50        1    1    6       30 0.2400745
% 95     50       10    2   10        5 0.2400896
% 464   500       10    3   31       30 0.2400896
% 119    50       10    5   10        5 0.2403832
% 455    50       10    2   31       30 0.2406993
% 463    50       10    3   31       30 0.2416290
% 472   500       10    4   31       30 0.2419188
% 121     1        1    1   31        5 0.2422275
% 99     50        1    3   10        5 0.2422388
% 125     1       10    1   31        5 0.2425361
% 266     5        1    4   10       10 0.2428711
% 240   500       10    5    6       10 0.2431572
% 282     5        1    1   31       10 0.2434621
% 212   500        1    2    6       10 0.2434734
% 471    50       10    4   31       30 0.2437669
% 274     5        1    5   10       10 0.2437669
% 447    50       10    1   31       30 0.2437820
% 224   500       10    3    6       10 0.2437858
% 129     1        1    2   31        5 0.2440229
% 286     5       10    1   31       10 0.2440605
% 208   500       10    1    6       10 0.2443917
% 115    50        1    5   10        5 0.2443917
% 220   500        1    3    6       10 0.2443955
% 133     1       10    2   31        5 0.2446326
% 426     5        1    4   10       30 0.2446740
% 239    50       10    5    6       10 0.2446966
% 232   500       10    4    6       10 0.2453252
% 216   500       10    2    6       10 0.2459463
% 228   500        1    4    6       10 0.2462549
% 223    50       10    3    6       10 0.2462624
% 231    50       10    4    6       10 0.2468722
% 236   500        1    5    6       10 0.2474782
% 262     5       10    3   10       10 0.2474895
% 204   500        1    1    6       10 0.2484003
% 235    50        1    5    6       10 0.2487127
% 450     5        1    2   31       30 0.2489875
% 278     5       10    5   10       10 0.2493263
% 137     1        1    3   31        5 0.2495822
% 141     1       10    3   31        5 0.2495822
% 153     1        1    5   31        5 0.2496236
% 157     1       10    5   31        5 0.2496236
% 258     5        1    3   10       10 0.2496236
% 434     5        1    5   10       30 0.2496274
% 474     5        1    5   31       30 0.2496349
% 442     5        1    1   31       30 0.2499285
% 402     5        1    1   10       30 0.2505571
% 215    50       10    2    6       10 0.2505721
% 203    50        1    1    6       10 0.2508732
% 227    50        1    4    6       10 0.2511856
% 466     5        1    4   31       30 0.2514604
% 207    50       10    1    6       10 0.2518029
% 211    50        1    2    6       10 0.2521266
% 82      5        1    1   10        5 0.2524089
% 219    50        1    3    6       10 0.2524127
% 270     5       10    4   10       10 0.2527176
% 86      5       10    1   10        5 0.2530375
% 118     5       10    5   10        5 0.2533348
% 106     5        1    4   10        5 0.2533424
% 110     5       10    4   10        5 0.2545694
% 114     5        1    5   10        5 0.2545694
% 145     1        1    4   31        5 0.2548442
% 394     5        1    5    6       30 0.2552130
% 386     5        1    4    6       30 0.2554464
% 149     1       10    4   31        5 0.2554577
% 378     5        1    3    6       30 0.2554577
% 250     5        1    2   10       10 0.2554765
% 458     5        1    3   31       30 0.2558077
% 414     5       10    2   10       30 0.2563987
% 246     5       10    1   10       10 0.2564100
% 438     5       10    5   10       30 0.2567186
% 90      5        1    2   10        5 0.2588904
% 297     1        1    3   31       10 0.2591614
% 362     5        1    1    6       30 0.2591915
% 406     5       10    1   10       30 0.2594964
% 94      5       10    2   10        5 0.2601250
% 210     5        1    2    6       10 0.2604110
% 234     5        1    5    6       10 0.2604675
% 418     5        1    3   10       30 0.2607385
% 374     5       10    2    6       30 0.2609944
% 301     1       10    3   31       10 0.2610132
% 202     5        1    1    6       10 0.2610547
% 305     1        1    4   31       10 0.2613257
% 430     5       10    4   10       30 0.2616569
% 242     5        1    1   10       10 0.2619655
% 470     5       10    4   31       30 0.2622591
% 254     5       10    2   10       10 0.2622742
% 382     5       10    3    6       30 0.2631813
% 446     5       10    1   31       30 0.2632039
% 70      5       10    4    6        5 0.2638362
% 454     5       10    2   31       30 0.2646944
% 422     5       10    3   10       30 0.2650256
% 390     5       10    4    6       30 0.2656165
% 214     5       10    2    6       10 0.2656542
% 410     5        1    2   10       30 0.2656542
% 478     5       10    5   31       30 0.2659515
% 309     1       10    4   31       10 0.2662639
% 285     1       10    1   31       10 0.2674947
% 238     5       10    5    6       10 0.2675248
% 230     5       10    4    6       10 0.2675474
% 281     1        1    1   31       10 0.2677808
% 66      5        1    4    6        5 0.2678561
% 58      5        1    3    6        5 0.2687669
% 277     1       10    5   10       10 0.2690078
% 366     5       10    1    6       30 0.2690492
% 206     5       10    1    6       10 0.2702913
% 222     5       10    3    6       10 0.2702913
% 370     5        1    2    6       30 0.2702951
% 43     50        1    1    6        5 0.2703177
% 226     5        1    4    6       10 0.2705774
% 63     50       10    3    6        5 0.2706113
% 257     1        1    3   10       10 0.2708898
% 398     5       10    5    6       30 0.2711871
% 313     1        1    5   31       10 0.2712173
% 218     5        1    3    6       10 0.2718195
% 62      5       10    3    6        5 0.2718534
% 102     5       10    3   10        5 0.2721808
% 293     1       10    2   31       10 0.2723615
% 98      5        1    3   10        5 0.2724744
% 273     1        1    5   10       10 0.2726965
% 462     5       10    3   31       30 0.2730616
% 317     1       10    5   31       10 0.2733815
% 289     1        1    2   31       10 0.2735735
% 47     50       10    1    6        5 0.2739913
% 59     50        1    3    6        5 0.2758657
% 71     50       10    4    6        5 0.2761668
% 51     50        1    2    6        5 0.2792307
% 79     50       10    5    6        5 0.2792419
% 261     1       10    3   10       10 0.2795054
% 64    500       10    3    6        5 0.2795581
% 48    500       10    1    6        5 0.2795619
% 117     1       10    5   10        5 0.2801265
% 113     1        1    5   10        5 0.2804351
% 74      5        1    5    6        5 0.2804765
% 56    500       10    2    6        5 0.2804765
% 72    500       10    4    6        5 0.2804803
% 75     50        1    5    6        5 0.2813987
% 60    500        1    3    6        5 0.2814100
% 52    500        1    2    6        5 0.2817148
% 67     50        1    4    6        5 0.2817224
% 68    500        1    4    6        5 0.2820235
% 76    500        1    5    6        5 0.2820310
% 44    500        1    1    6        5 0.2826332
% 80    500       10    5    6        5 0.2826332
% 55     50       10    2    6        5 0.2829569
% 78      5       10    5    6        5 0.2835629
% 42      5        1    1    6        5 0.2841840
% 105     1        1    4   10        5 0.2853922
% 46      5       10    1    6        5 0.2854073
% 109     1       10    4   10        5 0.2860020
% 465     1        1    4   31       30 0.2866004
% 101     1       10    3   10        5 0.2873043
% 97      1        1    3   10        5 0.2876092
% 334     5       10    2    1       30 0.2878802
% 330     5        1    2    1       30 0.2890846
% 358     5       10    5    1       30 0.2891260
% 457     1        1    3   31       30 0.2900068
% 69      1       10    4    6        5 0.2903117
% 65      1        1    4    6        5 0.2909327
% 50      5        1    2    6        5 0.2909553
% 54      5       10    2    6        5 0.2915876
% 354     5        1    5    1       30 0.2924910
% 437     1       10    5   10       30 0.2927394
% 473     1        1    5   31       30 0.2930631
% 322     5        1    1    1       30 0.2930857
% 335    50       10    2    1       30 0.2934319
% 327    50       10    1    1       30 0.2934319
% 186     5        1    4    1       10 0.2934357
% 441     1        1    1   31       30 0.2943127
% 225     1        1    4    6       10 0.2943202
% 190     5       10    4    1       10 0.2943428
% 77      1       10    5    6        5 0.2946176
% 57      1        1    3    6        5 0.2946477
% 346     5        1    4    1       30 0.2946665
% 73      1        1    5    6        5 0.2949262
% 343    50       10    3    1       30 0.2952763
% 433     1        1    5   10       30 0.2954871
% 350     5       10    4    1       30 0.2955586
% 339    50        1    3    1       30 0.2955887
% 347    50        1    4    1       30 0.2955924
% 170     5        1    2    1       10 0.2955962
% 61      1       10    3    6        5 0.2958785
% 162     5        1    1    1       10 0.2958823
% 198     5       10    5    1       10 0.2958898
% 323    50        1    1    1       30 0.2958973
% 425     1        1    4   10       30 0.2961834
% 194     5        1    5    1       10 0.2961909
% 326     5       10    1    1       30 0.2962060
% 331    50        1    2    1       30 0.2962060
% 344   500       10    3    1       30 0.2962060
% 348   500        1    4    1       30 0.2962060
% 166     5       10    1    1       10 0.2965108
% 182     5       10    3    1       10 0.2965221
% 355    50        1    5    1       30 0.2968157
% 2       5        1    1    1        5 0.2968232
% 324   500        1    1    1       30 0.2968232
% 328   500       10    1    1       30 0.2968232
% 340   500        1    3    1       30 0.2968232
% 6       5       10    1    1        5 0.2968232
% 233     1        1    5    6       10 0.2968232
% 352   500       10    4    1       30 0.2968232
% 360   500       10    5    1       30 0.2968232
% 359    50       10    5    1       30 0.2971281
% 332   500        1    2    1       30 0.2971319
% 336   500       10    2    1       30 0.2971319
% 356   500        1    5    1       30 0.2971319
% 26      5        1    4    1        5 0.2974368
% 351    50       10    4    1       30 0.2974368
% 3      50        1    1    1        5 0.2974405
% 4     500        1    1    1        5 0.2974405
% 7      50       10    1    1        5 0.2974405
% 8     500       10    1    1        5 0.2974405
% 11     50        1    2    1        5 0.2974405
% 12    500        1    2    1        5 0.2974405
% 15     50       10    2    1        5 0.2974405
% 16    500       10    2    1        5 0.2974405
% 19     50        1    3    1        5 0.2974405
% 20    500        1    3    1        5 0.2974405
% 22      5       10    3    1        5 0.2974405
% 23     50       10    3    1        5 0.2974405
% 24    500       10    3    1        5 0.2974405
% 27     50        1    4    1        5 0.2974405
% 28    500        1    4    1        5 0.2974405
% 31     50       10    4    1        5 0.2974405
% 32    500       10    4    1        5 0.2974405
% 35     50        1    5    1        5 0.2974405
% 36    500        1    5    1        5 0.2974405
% 39     50       10    5    1        5 0.2974405
% 40    500       10    5    1        5 0.2974405
% 163    50        1    1    1       10 0.2974405
% 164   500        1    1    1       10 0.2974405
% 167    50       10    1    1       10 0.2974405
% 168   500       10    1    1       10 0.2974405
% 171    50        1    2    1       10 0.2974405
% 172   500        1    2    1       10 0.2974405
% 175    50       10    2    1       10 0.2974405
% 176   500       10    2    1       10 0.2974405
% 179    50        1    3    1       10 0.2974405
% 180   500        1    3    1       10 0.2974405
% 183    50       10    3    1       10 0.2974405
% 184   500       10    3    1       10 0.2974405
% 187    50        1    4    1       10 0.2974405
% 188   500        1    4    1       10 0.2974405
% 191    50       10    4    1       10 0.2974405
% 192   500       10    4    1       10 0.2974405
% 195    50        1    5    1       10 0.2974405
% 196   500        1    5    1       10 0.2974405
% 199    50       10    5    1       10 0.2974405
% 200   500       10    5    1       10 0.2974405
% 237     1       10    5    6       10 0.2974443
% 10      5        1    2    1        5 0.2977492
% 14      5       10    2    1        5 0.2977492
% 34      5        1    5    1        5 0.2977605
% 38      5       10    5    1        5 0.2977605
% 174     5       10    2    1       10 0.2980465
% 18      5        1    3    1        5 0.2980540
% 342     5       10    3    1       30 0.2983439
% 30      5       10    4    1        5 0.2983627
% 241     1        1    1   10       10 0.2986488
% 178     5        1    3    1       10 0.2989724
% 221     1       10    3    6       10 0.2995634
% 229     1       10    4    6       10 0.2998645
% 338     5        1    3    1       30 0.3005345
% 245     1       10    1   10       10 0.3010765
% 401     1        1    1   10       30 0.3014152
% 13      1       10    2    1        5 0.3017389
% 29      1       10    4    1        5 0.3017502
% 269     1       10    4   10       10 0.3020137
% 9       1        1    2    1        5 0.3020476
% 217     1        1    3    6       10 0.3020551
% 449     1        1    2   31       30 0.3022998
% 265     1        1    4   10       10 0.3023449
% 89      1        1    2   10        5 0.3026423
% 93      1       10    2   10        5 0.3026423
% 85      1       10    1   10        5 0.3026498
% 81      1        1    1   10        5 0.3029547
% 253     1       10    2   10       10 0.3031843
% 417     1        1    3   10       30 0.3035757
% 1       1        1    1    1        5 0.3035945
% 25      1        1    4    1        5 0.3039107
% 173     1       10    2    1       10 0.3045092
% 385     1        1    4    6       30 0.3050512
% 461     1       10    3   31       30 0.3051076
% 5       1       10    1    1        5 0.3054464
% 249     1        1    2   10       10 0.3059621
% 169     1        1    2    1       10 0.3060035
% 389     1       10    4    6       30 0.3066659
% 49      1        1    2    6        5 0.3069557
% 53      1       10    2    6        5 0.3069557
% 165     1       10    1    1       10 0.3082167
% 33      1        1    5    1        5 0.3085328
% 37      1       10    5    1        5 0.3085328
% 161     1        1    1    1       10 0.3091388
% 193     1        1    5    1       10 0.3091652
% 393     1        1    5    6       30 0.3103245
% 197     1       10    5    1       10 0.3119429
% 189     1       10    4    1       10 0.3121876
% 381     1       10    3    6       30 0.3128162
% 185     1        1    4    1       10 0.3134410
% 353     1        1    5    1       30 0.3134937
% 469     1       10    4   31       30 0.3140432
% 21      1       10    3    1        5 0.3140696
% 377     1        1    3    6       30 0.3140808
% 429     1       10    4   10       30 0.3143255
% 17      1        1    3    1        5 0.3143744
% 397     1       10    5    6       30 0.3149466
% 421     1       10    3   10       30 0.3155751
% 453     1       10    2   31       30 0.3164521
% 209     1        1    2    6       10 0.3167871
% 409     1        1    2   10       30 0.3167871
% 445     1       10    1   31       30 0.3171183
% 345     1        1    4    1       30 0.3174533
% 413     1       10    2   10       30 0.3182739
% 333     1       10    2    1       30 0.3183416
% 213     1       10    2    6       10 0.3186352
% 321     1        1    1    1       30 0.3192675
% 329     1        1    2    1       30 0.3193127
% 405     1       10    1   10       30 0.3195310
% 201     1        1    1    6       10 0.3195912

\begin{verbatim}
361     1        1    1    6       30 0.3199074
477     1       10    5   31       30 0.3201558
177     1        1    3    1       10 0.3205360
369     1        1    2    6       30 0.3239235
45      1       10    1    6        5 0.3245220
357     1       10    5    1       30 0.3245897
41      1        1    1    6        5 0.3248269
205     1       10    1    6       10 0.3251167
181     1       10    3    1       10 0.3251618
349     1       10    4    1       30 0.3316057
365     1       10    1    6       30 0.3316095
325     1       10    1    1       30 0.3322418
337     1        1    3    1       30 0.3353395
373     1       10    2    6       30 0.3367773
341     1       10    3    1       30 0.3408913
\end{verbatim}

Results for math students
\begin{verbatim}
    ntree nodesize seed mtry maxnodes     error
136   500       10    2   31        5 0.2200255
147    50        1    4   31        5 0.2205063
151    50       10    4   31        5 0.2205357
160   500       10    5   31        5 0.2210165
131    50        1    2   31        5 0.2214678
148   500        1    4   31        5 0.2214973
132   500        1    2   31        5 0.2215267
124   500        1    1   31        5 0.2215856
128   500       10    1   31        5 0.2215856
140   500        1    3   31        5 0.2215856
152   500       10    4   31        5 0.2215856
135    50       10    2   31        5 0.2219780
143    50       10    3   31        5 0.2220663
144   500       10    3   31        5 0.2220663
159    50       10    5   31        5 0.2220663
127    50       10    1   31        5 0.2225765
156   500        1    5   31        5 0.2234498
139    50        1    3   31        5 0.2235086
155    50        1    5   31        5 0.2235381
123    50        1    1   31        5 0.2259125
130     5        1    2   31        5 0.2260302
134     5       10    2   31        5 0.2270506
158     5       10    5   31        5 0.2275903
154     5        1    5   31        5 0.2291209
145     1        1    4   31        5 0.2292975
149     1       10    4   31        5 0.2292975
247    50       10    1   10       10 0.2296311
291    50        1    2   31       10 0.2296900
315    50        1    5   31       10 0.2303179
...
\end{verbatim}

% 146     5        1    4   31        5 0.2310440
% 150     5       10    4   31        5 0.2315542
% 259    50        1    3   10       10 0.2315542
% 316   500        1    5   31       10 0.2317896
% 271    50       10    4   10       10 0.2322115
% 287    50       10    1   31       10 0.2326923
% 284   500        1    1   31       10 0.2327806
% 279    50       10    5   10       10 0.2330553
% 267    50        1    4   10       10 0.2331731
% 292   500        1    2   31       10 0.2332614
% 319    50       10    5   31       10 0.2332614
% 264   500       10    3   10       10 0.2332614
% 427    50        1    4   10       30 0.2337422
% 300   500        1    3   31       10 0.2338010
% 280   500       10    5   10       10 0.2341935
% 244   500        1    1   10       10 0.2342524
% 122     5        1    1   31        5 0.2343112
% 126     5       10    1   31        5 0.2343112
% 307    50        1    4   31       10 0.2343112
% 295    50       10    2   31       10 0.2346743
% 320   500       10    5   31       10 0.2348214
% 303    50       10    3   31       10 0.2348509
% 256   500       10    2   10       10 0.2351845
% 252   500        1    2   10       10 0.2352139
% 263    50       10    3   10       10 0.2352433
% 283    50        1    1   31       10 0.2352433
% 296   500       10    2   31       10 0.2353611
% 248   500       10    1   10       10 0.2357241
% 299    50        1    3   31       10 0.2358418
% 272   500       10    4   10       10 0.2362637
% 436   500        1    5   10       30 0.2363815
% 412   500        1    2   10       30 0.2367445
% 268   500        1    4   10       10 0.2367445
% 428   500        1    4   10       30 0.2367739
% 288   500       10    1   31       10 0.2368034
% 304   500       10    3   31       10 0.2368034
% 308   500        1    4   31       10 0.2368034
% 404   500        1    1   10       30 0.2368034
% 384   500       10    3    6       30 0.2371958
% 260   500        1    3   10       10 0.2371958
% 255    50       10    2   10       10 0.2372253
% 375    50       10    2    6       30 0.2373724
% 251    50        1    2   10       10 0.2376177
% 367    50       10    1    6       30 0.2376766
% 276   500        1    5   10       10 0.2377355
% 420   500        1    3   10       30 0.2378238
% 392   500       10    4    6       30 0.2381279
% 411    50        1    2   10       30 0.2381279
% 408   500       10    1   10       30 0.2382457
% 311    50       10    4   31       10 0.2383340
% 443    50        1    1   31       30 0.2383340
% 275    50        1    5   10       10 0.2386381
% 431    50       10    4   10       30 0.2387559
% 432   500       10    4   10       30 0.2388442
% 312   500       10    4   31       10 0.2388736
% 439    50       10    5   10       30 0.2392072
% 452   500        1    2   31       30 0.2393250
% 403    50        1    1   10       30 0.2394427
% 479    50       10    5   31       30 0.2396291
% 364   500        1    1    6       30 0.2396586
% 400   500       10    5    6       30 0.2396586
% 371    50        1    2    6       30 0.2397174
% 419    50        1    3   10       30 0.2397469
% 399    50       10    5    6       30 0.2397763
% 440   500       10    5   10       30 0.2398352
% 416   500       10    2   10       30 0.2398646
% 391    50       10    4    6       30 0.2398940
% 138     5        1    3   31        5 0.2400805
% 444   500        1    1   31       30 0.2403159
% 243    50        1    1   10       10 0.2408261
% 456   500       10    2   31       30 0.2408850
% 376   500       10    2    6       30 0.2411009
% 314     5        1    5   31       10 0.2411892
% 451    50        1    2   31       30 0.2413658
% 472   500       10    4   31       30 0.2413952
% 306     5        1    4   31       10 0.2414246
% 388   500        1    4    6       30 0.2416994
% 290     5        1    2   31       10 0.2417288
% 423    50       10    3   10       30 0.2418171
% 435    50        1    5   10       30 0.2418465
% 383    50       10    3    6       30 0.2419054
% 142     5       10    3   31        5 0.2421213
% 407    50       10    1   10       30 0.2422684
% 448   500       10    1   31       30 0.2424451
% 447    50       10    1   31       30 0.2425039
% 424   500       10    3   10       30 0.2427786
% 476   500        1    5   31       30 0.2428375
% 480   500       10    5   31       30 0.2428964
% 468   500        1    4   31       30 0.2433477
% 310     5       10    4   31       10 0.2434360
% 387    50        1    4    6       30 0.2437402
% 368   500       10    1    6       30 0.2437696
% 396   500        1    5    6       30 0.2437696
% 475    50        1    5   31       30 0.2437696
% 380   500        1    3    6       30 0.2442210
% 318     5       10    5   31       10 0.2442504
% 372   500        1    2    6       30 0.2447312
% 286     5       10    1   31       10 0.2448489
% 460   500        1    3   31       30 0.2448783
% 282     5        1    1   31       10 0.2454474
% 459    50        1    3   31       30 0.2458104
% 363    50        1    1    6       30 0.2458987
% 455    50       10    2   31       30 0.2459282
% 379    50        1    3    6       30 0.2462029
% 294     5       10    2   31       10 0.2463206
% 298     5        1    3   31       10 0.2463795
% 129     1        1    2   31        5 0.2465267
% 133     1       10    2   31        5 0.2465267
% 471    50       10    4   31       30 0.2469486
% 464   500       10    3   31       30 0.2473705
% 395    50        1    5    6       30 0.2477630
% 467    50        1    4   31       30 0.2479396
% 223    50       10    3    6       10 0.2482143
% 216   500       10    2    6       10 0.2482437
% 418     5        1    3   10       30 0.2484498
% 302     5       10    3   31       10 0.2484792
% 231    50       10    4    6       10 0.2497449
% 240   500       10    5    6       10 0.2502551
% 211    50        1    2    6       10 0.2507064
% 463    50       10    3   31       30 0.2510302
% 235    50        1    5    6       10 0.2511578
% 120   500       10    5   10        5 0.2513049
% 270     5       10    4   10       10 0.2516582
% 232   500       10    4    6       10 0.2517857
% 415    50       10    2   10       30 0.2520801
% 219    50        1    3    6       10 0.2521782
% 115    50        1    5   10        5 0.2522370
% 99     50        1    3   10        5 0.2522959
% 239    50       10    5    6       10 0.2522959
% 218     5        1    3    6       10 0.2523842
% 96    500       10    2   10        5 0.2527767
% 224   500       10    3    6       10 0.2528061
% 119    50       10    5   10        5 0.2528650
% 87     50       10    1   10        5 0.2532575
% 208   500       10    1    6       10 0.2532575
% 88    500       10    1   10        5 0.2532869
% 204   500        1    1    6       10 0.2532869
% 104   500       10    3   10        5 0.2533163
% 95     50       10    2   10        5 0.2537088
% 103    50       10    3   10        5 0.2537088
% 207    50       10    1    6       10 0.2537677
% 236   500        1    5    6       10 0.2537971
% 242     5        1    1   10       10 0.2538265
% 91     50        1    2   10        5 0.2538560
% 250     5        1    2   10       10 0.2541503
% 111    50       10    4   10        5 0.2542190
% 83     50        1    1   10        5 0.2546998
% 212   500        1    2    6       10 0.2547881
% 203    50        1    1    6       10 0.2548764
% 100   500        1    3   10        5 0.2553571
% 246     5       10    1   10       10 0.2554454
% 84    500        1    1   10        5 0.2557496
% 90      5        1    2   10        5 0.2561911
% 116   500        1    5   10        5 0.2562304
% 227    50        1    4    6       10 0.2562892
% 112   500       10    4   10        5 0.2563187
% 215    50       10    2    6       10 0.2563481
% 220   500        1    3    6       10 0.2563481
% 82      5        1    1   10        5 0.2564659
% 474     5        1    5   31       30 0.2566425
% 228   500        1    4    6       10 0.2573685
% 234     5        1    5    6       10 0.2575451
% 108   500        1    4   10        5 0.2578787
% 222     5       10    3    6       10 0.2578787
% 206     5       10    1    6       10 0.2582614
% 442     5        1    1   31       30 0.2586538
% 258     5        1    3   10       10 0.2590757
% 202     5        1    1    6       10 0.2592818
% 254     5       10    2   10       10 0.2594976
% 305     1        1    4   31       10 0.2594976
% 94      5       10    2   10        5 0.2597037
% 92    500        1    2   10        5 0.2598607
% 466     5        1    4   31       30 0.2602433
% 86      5       10    1   10        5 0.2605181
% 114     5        1    5   10        5 0.2605769
% 434     5        1    5   10       30 0.2609105
% 266     5        1    4   10       10 0.2611754
% 262     5       10    3   10       10 0.2613913
% 402     5        1    1   10       30 0.2614403
% 309     1       10    4   31       10 0.2615385
% 121     1        1    1   31        5 0.2624706
% 107    50        1    4   10        5 0.2625294
% 118     5       10    5   10        5 0.2625294
% 125     1       10    1   31        5 0.2629808
% 278     5       10    5   10       10 0.2629808
% 446     5       10    1   31       30 0.2643838
% 458     5        1    3   31       30 0.2656201
% 289     1        1    2   31       10 0.2659537
% 293     1       10    2   31       10 0.2660126
% 410     5        1    2   10       30 0.2675726
% 426     5        1    4   10       30 0.2676903
% 274     5        1    5   10       10 0.2677492
% 153     1        1    5   31        5 0.2680534
% 106     5        1    4   10        5 0.2682594
% 214     5       10    2    6       10 0.2682594
% 374     5       10    2    6       30 0.2683477
% 157     1       10    5   31        5 0.2685636
% 210     5        1    2    6       10 0.2686519
% 226     5        1    4    6       10 0.2690443
% 42      5        1    1    6        5 0.2700648
% 110     5       10    4   10        5 0.2712912
% 378     5        1    3    6       30 0.2714973
% 398     5       10    5    6       30 0.2715365
% 46      5       10    1    6        5 0.2716248
% 462     5       10    3   31       30 0.2734203
% 430     5       10    4   10       30 0.2744702
% 478     5       10    5   31       30 0.2748038
% 406     5       10    1   10       30 0.2752845
% 370     5        1    2    6       30 0.2758536
% 43     50        1    1    6        5 0.2760989
% 414     5       10    2   10       30 0.2762755
% 285     1       10    1   31       10 0.2763344
% 386     5        1    4    6       30 0.2763638
% 47     50       10    1    6        5 0.2764914
% 78      5       10    5    6        5 0.2765502
% 394     5        1    5    6       30 0.2768151
% 55     50       10    2    6        5 0.2770310
% 54      5       10    2    6        5 0.2771193
% 102     5       10    3   10        5 0.2772959
% 51     50        1    2    6        5 0.2780220
% 245     1       10    1   10       10 0.2783163
% 79     50       10    5    6        5 0.2785027
% 450     5        1    2   31       30 0.2787088
% 382     5       10    3    6       30 0.2788854
% 59     50        1    3    6        5 0.2789835
% 454     5       10    2   31       30 0.2791013
% 71     50       10    4    6        5 0.2796409
% 74      5        1    5    6        5 0.2801805
% 281     1        1    1   31       10 0.2803571
% 313     1        1    5   31       10 0.2806613
% 50      5        1    2    6        5 0.2806907
% 48    500       10    1    6        5 0.2816228
% 45      1       10    1    6        5 0.2816523
% 98      5        1    3   10        5 0.2822508
% 233     1        1    5    6       10 0.2828493
% 64    500       10    3    6        5 0.2831240
% 230     5       10    4    6       10 0.2833301
% 80    500       10    5    6        5 0.2836342
% 52    500        1    2    6        5 0.2836342
% 362     5        1    1    6       30 0.2839874
% 56    500       10    2    6        5 0.2841444
% 41      1        1    1    6        5 0.2842622
% 390     5       10    4    6       30 0.2845271
% 67     50        1    4    6        5 0.2851648
% 238     5       10    5    6       10 0.2852531
% 342     5       10    3    1       30 0.2852531
% 241     1        1    1   10       10 0.2854297
% 470     5       10    4   31       30 0.2857633
% 63     50       10    3    6        5 0.2866366
% 44    500        1    1    6        5 0.2866954
% 72    500       10    4    6        5 0.2866954
% 317     1       10    5   31       10 0.2867838
% 62      5       10    3    6        5 0.2872645
% 70      5       10    4    6        5 0.2874706
% 438     5       10    5   10       30 0.2874706
% 76    500        1    5    6        5 0.2876864
% 75     50        1    5    6        5 0.2882261
% 422     5       10    3   10       30 0.2882555
% 330     5        1    2    1       30 0.2882849
% 66      5        1    4    6        5 0.2884615
% 166     5       10    1    1       10 0.2887363
% 58      5        1    3    6        5 0.2887951
% 60    500        1    3    6        5 0.2891876
% 297     1        1    3   31       10 0.2902963
% 137     1        1    3   31        5 0.2904435
% 343    50       10    3    1       30 0.2907182
% 68    500        1    4    6        5 0.2907476
% 366     5       10    1    6       30 0.2907771
% 359    50       10    5    1       30 0.2912578
% 237     1       10    5    6       10 0.2915816
% 186     5        1    4    1       10 0.2917975
% 162     5        1    1    1       10 0.2937206
% 327    50       10    1    1       30 0.2937500
% 141     1       10    3   31        5 0.2945251
% 355    50        1    5    1       30 0.2947410
% 351    50       10    4    1       30 0.2947704
% 323    50        1    1    1       30 0.2947998
% 331    50        1    2    1       30 0.2947998
% 81      1        1    1   10        5 0.2950353
% 85      1       10    1   10        5 0.2950353
% 174     5       10    2    1       10 0.2953984
% 441     1        1    1   31       30 0.2955455
% 170     5        1    2    1       10 0.2957614
% 34      5        1    5    1        5 0.2957614
% 178     5        1    3    1       10 0.2957908
% 335    50       10    2    1       30 0.2963010
% 350     5       10    4    1       30 0.2963010
% 326     5       10    1    1       30 0.2963893
% 334     5       10    2    1       30 0.2964776
% 322     5        1    1    1       30 0.2968112
% 249     1        1    2   10       10 0.2968603
% 347    50        1    4    1       30 0.2972920
% 38      5       10    5    1        5 0.2973214
% 344   500       10    3    1       30 0.2973214
% 360   500       10    5    1       30 0.2973214
% 269     1       10    4   10       10 0.2974294
% 198     5       10    5    1       10 0.2977433
% 358     5       10    5    1       30 0.2977728
% 339    50        1    3    1       30 0.2978022
% 328   500       10    1    1       30 0.2978316
% 336   500       10    2    1       30 0.2978316
% 352   500       10    4    1       30 0.2978316
% 93      1       10    2   10        5 0.2981849
% 348   500        1    4    1       30 0.2983418
% 445     1       10    1   31       30 0.2987834
% 26      5        1    4    1        5 0.2988226
% 301     1       10    3   31       10 0.2988520
% 265     1        1    4   10       10 0.2989600
% 2       5        1    1    1        5 0.2993328
% 324   500        1    1    1       30 0.2993328
% 332   500        1    2    1       30 0.2993328
% 356   500        1    5    1       30 0.2993328
% 14      5       10    2    1        5 0.2993328
% 18      5        1    3    1        5 0.2993328
% 22      5       10    3    1        5 0.2993622
% 10      5        1    2    1        5 0.2993917
% 253     1       10    2   10       10 0.2994996
% 6       5       10    1    1        5 0.2998430
% 190     5       10    4    1       10 0.2998430
% 340   500        1    3    1       30 0.2998430
% 89      1        1    2   10        5 0.3001079
% 182     5       10    3    1       10 0.3003238
% 175    50       10    2    1       10 0.3003532
% 171    50        1    2    1       10 0.3003532
% 191    50       10    4    1       10 0.3003532
% 30      5       10    4    1        5 0.3008634
% 3      50        1    1    1        5 0.3008634
% 4     500        1    1    1        5 0.3008634
% 7      50       10    1    1        5 0.3008634
% 8     500       10    1    1        5 0.3008634
% 11     50        1    2    1        5 0.3008634
% 12    500        1    2    1        5 0.3008634
% 15     50       10    2    1        5 0.3008634
% 16    500       10    2    1        5 0.3008634
% 19     50        1    3    1        5 0.3008634
% 20    500        1    3    1        5 0.3008634
% 23     50       10    3    1        5 0.3008634
% 24    500       10    3    1        5 0.3008634
% 27     50        1    4    1        5 0.3008634
% 28    500        1    4    1        5 0.3008634
% 31     50       10    4    1        5 0.3008634
% 32    500       10    4    1        5 0.3008634
% 35     50        1    5    1        5 0.3008634
% 36    500        1    5    1        5 0.3008634
% 39     50       10    5    1        5 0.3008634
% 40    500       10    5    1        5 0.3008634
% 163    50        1    1    1       10 0.3008634
% 164   500        1    1    1       10 0.3008634
% 167    50       10    1    1       10 0.3008634
% 168   500       10    1    1       10 0.3008634
% 172   500        1    2    1       10 0.3008634
% 176   500       10    2    1       10 0.3008634
% 179    50        1    3    1       10 0.3008634
% 180   500        1    3    1       10 0.3008634
% 183    50       10    3    1       10 0.3008634
% 184   500       10    3    1       10 0.3008634
% 187    50        1    4    1       10 0.3008634
% 188   500        1    4    1       10 0.3008634
% 192   500       10    4    1       10 0.3008634
% 195    50        1    5    1       10 0.3008634
% 196   500        1    5    1       10 0.3008634
% 199    50       10    5    1       10 0.3008634
% 200   500       10    5    1       10 0.3008634
% 205     1       10    1    6       10 0.3025412
% 338     5        1    3    1       30 0.3031397
% 105     1        1    4   10        5 0.3032869
% 257     1        1    3   10       10 0.3053964
% 354     5        1    5    1       30 0.3054258
% 346     5        1    4    1       30 0.3055141
% 194     5        1    5    1       10 0.3060538
% 17      1        1    3    1        5 0.3065051
% 109     1       10    4   10        5 0.3068583
% 201     1        1    1    6       10 0.3071919
% 9       1        1    2    1        5 0.3079768
% 21      1       10    3    1        5 0.3080357
% 97      1        1    3   10        5 0.3084870
% 173     1       10    2    1       10 0.3085165
% 73      1        1    5    6        5 0.3087225
% 77      1       10    5    6        5 0.3087225
% 197     1       10    5    1       10 0.3089384
% 13      1       10    2    1        5 0.3089973
% 5       1       10    1    1        5 0.3090267
% 169     1        1    2    1       10 0.3097135
% 61      1       10    3    6        5 0.3101648
% 217     1        1    3    6       10 0.3103414
% 1       1        1    1    1        5 0.3105573
% 57      1        1    3    6        5 0.3106456
% 161     1        1    1    1       10 0.3120879
% 181     1       10    3    1       10 0.3121173
% 473     1        1    5   31       30 0.3122351
% 465     1        1    4   31       30 0.3132261
% 101     1       10    3   10        5 0.3135891
% 165     1       10    1    1       10 0.3136480
% 273     1        1    5   10       10 0.3138246
% 341     1       10    3    1       30 0.3139521
% 113     1        1    5   10        5 0.3143053
% 117     1       10    5   10        5 0.3143053
% 53      1       10    2    6        5 0.3146978
% 49      1        1    2    6        5 0.3152080
% 177     1        1    3    1       10 0.3156005
% 405     1       10    1   10       30 0.3160714
% 209     1        1    2    6       10 0.3161303
% 377     1        1    3    6       30 0.3170035
% 261     1       10    3   10       10 0.3171605
% 417     1        1    3   10       30 0.3174549
% 221     1       10    3    6       10 0.3177590
% 369     1        1    2    6       30 0.3182005
% 33      1        1    5    1        5 0.3186323
% 37      1       10    5    1        5 0.3186323
% 185     1        1    4    1       10 0.3186911
% 477     1       10    5   31       30 0.3194074
% 401     1        1    1   10       30 0.3195546
% 225     1        1    4    6       10 0.3204278
% 213     1       10    2    6       10 0.3207810
% 333     1       10    2    1       30 0.3213893
% 277     1       10    5   10       10 0.3214482
% 425     1        1    4   10       30 0.3215659
% 353     1        1    5    1       30 0.3216641
% 409     1        1    2   10       30 0.3221350
% 25      1        1    4    1        5 0.3232830
% 29      1       10    4    1        5 0.3232830
% 469     1       10    4   31       30 0.3234007
% 413     1       10    2   10       30 0.3238815
% 357     1       10    5    1       30 0.3241268
% 449     1        1    2   31       30 0.3248724
% 433     1        1    5   10       30 0.3249608
% 229     1       10    4    6       10 0.3255298
% 193     1        1    5    1       10 0.3256279
% 189     1       10    4    1       10 0.3268544
% 457     1        1    3   31       30 0.3270310
% 345     1        1    4    1       30 0.3270703
% 321     1        1    1    1       30 0.3270997
% 361     1        1    1    6       30 0.3303277
% 453     1       10    2   31       30 0.3309360
% 65      1        1    4    6        5 0.3316228
% 69      1       10    4    6        5 0.3326432
% 325     1       10    1    1       30 0.3327414

\begin{verbatim}
393     1        1    5    6       30 0.3341739
397     1       10    5    6       30 0.3343897
389     1       10    4    6       30 0.3344192
385     1        1    4    6       30 0.3348999
337     1        1    3    1       30 0.3353218
329     1        1    2    1       30 0.3372645
461     1       10    3   31       30 0.3381083
365     1       10    1    6       30 0.3445644
421     1       10    3   10       30 0.3486754
429     1       10    4   10       30 0.3498430
381     1       10    3    6       30 0.3517955
437     1       10    5   10       30 0.3529042
373     1       10    2    6       30 0.3552394
349     1       10    4    1       30 0.3581044
\end{verbatim}


Results for portugese language students:
\begin{verbatim}
    ntree nodesize seed mtry maxnodes     error
124   500        1    1   31        5 0.2234380
128   500       10    1   31        5 0.2234380
132   500        1    2   31        5 0.2234380
140   500        1    3   31        5 0.2234380
144   500       10    3   31        5 0.2234380
147    50        1    4   31        5 0.2234380
148   500        1    4   31        5 0.2234380
152   500       10    4   31        5 0.2234380
156   500        1    5   31        5 0.2234380
160   500       10    5   31        5 0.2234380
131    50        1    2   31        5 0.2237466
135    50       10    2   31        5 0.2237466
136   500       10    2   31        5 0.2237466
151    50       10    4   31        5 0.2237466
155    50        1    5   31        5 0.2237466
159    50       10    5   31        5 0.2237466
130     5        1    2   31        5 0.2240477
127    50       10    1   31        5 0.2240553
154     5        1    5   31        5 0.2240553
158     5       10    5   31        5 0.2240553
143    50       10    3   31        5 0.2243601
123    50        1    1   31        5 0.2243639
134     5       10    2   31        5 0.2246575
139    50        1    3   31        5 0.2246688
432   500       10    4   10       30 0.2252748
404   500        1    1   10       30 0.2255646
420   500        1    3   10       30 0.2255759
412   500        1    2   10       30 0.2258732
440   500       10    5   10       30 0.2258770
436   500        1    5   10       30 0.2258808
424   500       10    3   10       30 0.2261932
146     5        1    4   31        5 0.2268255
150     5       10    4   31        5 0.2268255
252   500        1    2   10       10 0.2271304
...
\end{verbatim}

% 260   500        1    3   10       10 0.2271304
% 268   500        1    4   10       10 0.2274390
% 428   500        1    4   10       30 0.2277326
% 264   500       10    3   10       10 0.2277477
% 291    50        1    2   31       10 0.2277552
% 248   500       10    1   10       10 0.2280601
% 280   500       10    5   10       10 0.2280601
% 408   500       10    1   10       30 0.2283461
% 431    50       10    4   10       30 0.2283499
% 272   500       10    4   10       10 0.2283650
% 416   500       10    2   10       30 0.2289559
% 276   500        1    5   10       10 0.2289785
% 122     5        1    1   31        5 0.2289860
% 126     5       10    1   31        5 0.2289898
% 251    50        1    2   10       10 0.2289898
% 244   500        1    1   10       10 0.2292871
% 435    50        1    5   10       30 0.2295845
% 138     5        1    3   31        5 0.2295882
% 142     5       10    3   31        5 0.2295882
% 419    50        1    3   10       30 0.2295958
% 256   500       10    2   10       10 0.2295958
% 403    50        1    1   10       30 0.2298856
% 415    50       10    2   10       30 0.2298856
% 271    50       10    4   10       10 0.2299044
% 423    50       10    3   10       30 0.2301942
% 307    50        1    4   31       10 0.2302243
% 295    50       10    2   31       10 0.2302281
% 384   500       10    3    6       30 0.2305104
% 292   500        1    2   31       10 0.2305367
% 467    50        1    4   31       30 0.2308266
% 375    50       10    2    6       30 0.2311201
% 407    50       10    1   10       30 0.2311277
% 314     5        1    5   31       10 0.2311390
% 308   500        1    4   31       10 0.2311540
% 263    50       10    3   10       10 0.2314363
% 267    50        1    4   10       10 0.2314514
% 283    50        1    1   31       10 0.2314551
% 311    50       10    4   31       10 0.2314589
% 439    50       10    5   10       30 0.2317337
% 255    50       10    2   10       10 0.2317638
% 304   500       10    3   31       10 0.2317675
% 296   500       10    2   31       10 0.2317713
% 379    50        1    3    6       30 0.2320536
% 275    50        1    5   10       10 0.2320574
% 279    50       10    5   10       10 0.2320574
% 303    50       10    3   31       10 0.2320724
% 312   500       10    4   31       10 0.2320762
% 316   500        1    5   31       10 0.2320799
% 259    50        1    3   10       10 0.2323622
% 298     5        1    3   31       10 0.2323622
% 243    50        1    1   10       10 0.2323660
% 319    50       10    5   31       10 0.2323773
% 287    50       10    1   31       10 0.2323773
% 315    50        1    5   31       10 0.2323848
% 320   500       10    5   31       10 0.2323886
% 427    50        1    4   10       30 0.2326558
% 411    50        1    2   10       30 0.2326671
% 247    50       10    1   10       10 0.2326746
% 300   500        1    3   31       10 0.2326972
% 288   500       10    1   31       10 0.2326972
% 376   500       10    2    6       30 0.2329569
% 88    500       10    1   10        5 0.2329908
% 92    500        1    2   10        5 0.2329908
% 100   500        1    3   10        5 0.2329946
% 400   500       10    5    6       30 0.2332731
% 372   500        1    2    6       30 0.2332919
% 104   500       10    3   10        5 0.2332957
% 310     5       10    4   31       10 0.2332957
% 475    50        1    5   31       30 0.2333107
% 318     5       10    5   31       10 0.2333107
% 284   500        1    1   31       10 0.2333145
% 299    50        1    3   31       10 0.2336006
% 451    50        1    2   31       30 0.2336081
% 476   500        1    5   31       30 0.2336119
% 388   500        1    4    6       30 0.2338979
% 380   500        1    3    6       30 0.2338979
% 392   500       10    4    6       30 0.2339017
% 396   500        1    5    6       30 0.2339055
% 116   500        1    5   10        5 0.2339167
% 306     5        1    4   31       10 0.2342103
% 460   500        1    3   31       30 0.2342216
% 84    500        1    1   10        5 0.2342254
% 96    500       10    2   10        5 0.2342291
% 395    50        1    5    6       30 0.2345303
% 112   500       10    4   10        5 0.2345340
% 452   500        1    2   31       30 0.2345340
% 383    50       10    3    6       30 0.2348389
% 459    50        1    3   31       30 0.2348389
% 108   500        1    4   10        5 0.2348427
% 444   500        1    1   31       30 0.2348427
% 120   500       10    5   10        5 0.2348464
% 468   500        1    4   31       30 0.2351588
% 368   500       10    1    6       30 0.2357498
% 364   500        1    1    6       30 0.2357498
% 443    50        1    1   31       30 0.2357573
% 399    50       10    5    6       30 0.2363633
% 391    50       10    4    6       30 0.2366795
% 302     5       10    3   31       10 0.2369919
% 103    50       10    3   10        5 0.2369956
% 456   500       10    2   31       30 0.2372930
% 83     50        1    1   10        5 0.2373005
% 371    50        1    2    6       30 0.2373193
% 480   500       10    5   31       30 0.2376092
% 367    50       10    1    6       30 0.2379065
% 111    50       10    4   10        5 0.2379178
% 107    50        1    4   10        5 0.2379291
% 448   500       10    1   31       30 0.2382340
% 87     50       10    1   10        5 0.2391561
% 387    50        1    4    6       30 0.2394572
% 294     5       10    2   31       10 0.2394723
% 290     5        1    2   31       10 0.2397696
% 479    50       10    5   31       30 0.2397696
% 91     50        1    2   10        5 0.2397734
% 363    50        1    1    6       30 0.2400745
% 95     50       10    2   10        5 0.2400896
% 464   500       10    3   31       30 0.2400896
% 119    50       10    5   10        5 0.2403832
% 455    50       10    2   31       30 0.2406993
% 463    50       10    3   31       30 0.2416290
% 472   500       10    4   31       30 0.2419188
% 121     1        1    1   31        5 0.2422275
% 99     50        1    3   10        5 0.2422388
% 125     1       10    1   31        5 0.2425361
% 266     5        1    4   10       10 0.2428711
% 240   500       10    5    6       10 0.2431572
% 282     5        1    1   31       10 0.2434621
% 212   500        1    2    6       10 0.2434734
% 471    50       10    4   31       30 0.2437669
% 274     5        1    5   10       10 0.2437669
% 447    50       10    1   31       30 0.2437820
% 224   500       10    3    6       10 0.2437858
% 129     1        1    2   31        5 0.2440229
% 286     5       10    1   31       10 0.2440605
% 208   500       10    1    6       10 0.2443917
% 115    50        1    5   10        5 0.2443917
% 220   500        1    3    6       10 0.2443955
% 133     1       10    2   31        5 0.2446326
% 426     5        1    4   10       30 0.2446740
% 239    50       10    5    6       10 0.2446966
% 232   500       10    4    6       10 0.2453252
% 216   500       10    2    6       10 0.2459463
% 228   500        1    4    6       10 0.2462549
% 223    50       10    3    6       10 0.2462624
% 231    50       10    4    6       10 0.2468722
% 236   500        1    5    6       10 0.2474782
% 262     5       10    3   10       10 0.2474895
% 204   500        1    1    6       10 0.2484003
% 235    50        1    5    6       10 0.2487127
% 450     5        1    2   31       30 0.2489875
% 278     5       10    5   10       10 0.2493263
% 137     1        1    3   31        5 0.2495822
% 141     1       10    3   31        5 0.2495822
% 153     1        1    5   31        5 0.2496236
% 157     1       10    5   31        5 0.2496236
% 258     5        1    3   10       10 0.2496236
% 434     5        1    5   10       30 0.2496274
% 474     5        1    5   31       30 0.2496349
% 442     5        1    1   31       30 0.2499285
% 402     5        1    1   10       30 0.2505571
% 215    50       10    2    6       10 0.2505721
% 203    50        1    1    6       10 0.2508732
% 227    50        1    4    6       10 0.2511856
% 466     5        1    4   31       30 0.2514604
% 207    50       10    1    6       10 0.2518029
% 211    50        1    2    6       10 0.2521266
% 82      5        1    1   10        5 0.2524089
% 219    50        1    3    6       10 0.2524127
% 270     5       10    4   10       10 0.2527176
% 86      5       10    1   10        5 0.2530375
% 118     5       10    5   10        5 0.2533348
% 106     5        1    4   10        5 0.2533424
% 110     5       10    4   10        5 0.2545694
% 114     5        1    5   10        5 0.2545694
% 145     1        1    4   31        5 0.2548442
% 394     5        1    5    6       30 0.2552130
% 386     5        1    4    6       30 0.2554464
% 149     1       10    4   31        5 0.2554577
% 378     5        1    3    6       30 0.2554577
% 250     5        1    2   10       10 0.2554765
% 458     5        1    3   31       30 0.2558077
% 414     5       10    2   10       30 0.2563987
% 246     5       10    1   10       10 0.2564100
% 438     5       10    5   10       30 0.2567186
% 90      5        1    2   10        5 0.2588904
% 297     1        1    3   31       10 0.2591614
% 362     5        1    1    6       30 0.2591915
% 406     5       10    1   10       30 0.2594964
% 94      5       10    2   10        5 0.2601250
% 210     5        1    2    6       10 0.2604110
% 234     5        1    5    6       10 0.2604675
% 418     5        1    3   10       30 0.2607385
% 374     5       10    2    6       30 0.2609944
% 301     1       10    3   31       10 0.2610132
% 202     5        1    1    6       10 0.2610547
% 305     1        1    4   31       10 0.2613257
% 430     5       10    4   10       30 0.2616569
% 242     5        1    1   10       10 0.2619655
% 470     5       10    4   31       30 0.2622591
% 254     5       10    2   10       10 0.2622742
% 382     5       10    3    6       30 0.2631813
% 446     5       10    1   31       30 0.2632039
% 70      5       10    4    6        5 0.2638362
% 454     5       10    2   31       30 0.2646944
% 422     5       10    3   10       30 0.2650256
% 390     5       10    4    6       30 0.2656165
% 214     5       10    2    6       10 0.2656542
% 410     5        1    2   10       30 0.2656542
% 478     5       10    5   31       30 0.2659515
% 309     1       10    4   31       10 0.2662639
% 285     1       10    1   31       10 0.2674947
% 238     5       10    5    6       10 0.2675248
% 230     5       10    4    6       10 0.2675474
% 281     1        1    1   31       10 0.2677808
% 66      5        1    4    6        5 0.2678561
% 58      5        1    3    6        5 0.2687669
% 277     1       10    5   10       10 0.2690078
% 366     5       10    1    6       30 0.2690492
% 206     5       10    1    6       10 0.2702913
% 222     5       10    3    6       10 0.2702913
% 370     5        1    2    6       30 0.2702951
% 43     50        1    1    6        5 0.2703177
% 226     5        1    4    6       10 0.2705774
% 63     50       10    3    6        5 0.2706113
% 257     1        1    3   10       10 0.2708898
% 398     5       10    5    6       30 0.2711871
% 313     1        1    5   31       10 0.2712173
% 218     5        1    3    6       10 0.2718195
% 62      5       10    3    6        5 0.2718534
% 102     5       10    3   10        5 0.2721808
% 293     1       10    2   31       10 0.2723615
% 98      5        1    3   10        5 0.2724744
% 273     1        1    5   10       10 0.2726965
% 462     5       10    3   31       30 0.2730616
% 317     1       10    5   31       10 0.2733815
% 289     1        1    2   31       10 0.2735735
% 47     50       10    1    6        5 0.2739913
% 59     50        1    3    6        5 0.2758657
% 71     50       10    4    6        5 0.2761668
% 51     50        1    2    6        5 0.2792307
% 79     50       10    5    6        5 0.2792419
% 261     1       10    3   10       10 0.2795054
% 64    500       10    3    6        5 0.2795581
% 48    500       10    1    6        5 0.2795619
% 117     1       10    5   10        5 0.2801265
% 113     1        1    5   10        5 0.2804351
% 74      5        1    5    6        5 0.2804765
% 56    500       10    2    6        5 0.2804765
% 72    500       10    4    6        5 0.2804803
% 75     50        1    5    6        5 0.2813987
% 60    500        1    3    6        5 0.2814100
% 52    500        1    2    6        5 0.2817148
% 67     50        1    4    6        5 0.2817224
% 68    500        1    4    6        5 0.2820235
% 76    500        1    5    6        5 0.2820310
% 44    500        1    1    6        5 0.2826332
% 80    500       10    5    6        5 0.2826332
% 55     50       10    2    6        5 0.2829569
% 78      5       10    5    6        5 0.2835629
% 42      5        1    1    6        5 0.2841840
% 105     1        1    4   10        5 0.2853922
% 46      5       10    1    6        5 0.2854073
% 109     1       10    4   10        5 0.2860020
% 465     1        1    4   31       30 0.2866004
% 101     1       10    3   10        5 0.2873043
% 97      1        1    3   10        5 0.2876092
% 334     5       10    2    1       30 0.2878802
% 330     5        1    2    1       30 0.2890846
% 358     5       10    5    1       30 0.2891260
% 457     1        1    3   31       30 0.2900068
% 69      1       10    4    6        5 0.2903117
% 65      1        1    4    6        5 0.2909327
% 50      5        1    2    6        5 0.2909553
% 54      5       10    2    6        5 0.2915876
% 354     5        1    5    1       30 0.2924910
% 437     1       10    5   10       30 0.2927394
% 473     1        1    5   31       30 0.2930631
% 322     5        1    1    1       30 0.2930857
% 335    50       10    2    1       30 0.2934319
% 327    50       10    1    1       30 0.2934319
% 186     5        1    4    1       10 0.2934357
% 441     1        1    1   31       30 0.2943127
% 225     1        1    4    6       10 0.2943202
% 190     5       10    4    1       10 0.2943428
% 77      1       10    5    6        5 0.2946176
% 57      1        1    3    6        5 0.2946477
% 346     5        1    4    1       30 0.2946665
% 73      1        1    5    6        5 0.2949262
% 343    50       10    3    1       30 0.2952763
% 433     1        1    5   10       30 0.2954871
% 350     5       10    4    1       30 0.2955586
% 339    50        1    3    1       30 0.2955887
% 347    50        1    4    1       30 0.2955924
% 170     5        1    2    1       10 0.2955962
% 61      1       10    3    6        5 0.2958785
% 162     5        1    1    1       10 0.2958823
% 198     5       10    5    1       10 0.2958898
% 323    50        1    1    1       30 0.2958973
% 425     1        1    4   10       30 0.2961834
% 194     5        1    5    1       10 0.2961909
% 326     5       10    1    1       30 0.2962060
% 331    50        1    2    1       30 0.2962060
% 344   500       10    3    1       30 0.2962060
% 348   500        1    4    1       30 0.2962060
% 166     5       10    1    1       10 0.2965108
% 182     5       10    3    1       10 0.2965221
% 355    50        1    5    1       30 0.2968157
% 2       5        1    1    1        5 0.2968232
% 324   500        1    1    1       30 0.2968232
% 328   500       10    1    1       30 0.2968232
% 340   500        1    3    1       30 0.2968232
% 6       5       10    1    1        5 0.2968232
% 233     1        1    5    6       10 0.2968232
% 352   500       10    4    1       30 0.2968232
% 360   500       10    5    1       30 0.2968232
% 359    50       10    5    1       30 0.2971281
% 332   500        1    2    1       30 0.2971319
% 336   500       10    2    1       30 0.2971319
% 356   500        1    5    1       30 0.2971319
% 26      5        1    4    1        5 0.2974368
% 351    50       10    4    1       30 0.2974368
% 3      50        1    1    1        5 0.2974405
% 4     500        1    1    1        5 0.2974405
% 7      50       10    1    1        5 0.2974405
% 8     500       10    1    1        5 0.2974405
% 11     50        1    2    1        5 0.2974405
% 12    500        1    2    1        5 0.2974405
% 15     50       10    2    1        5 0.2974405
% 16    500       10    2    1        5 0.2974405
% 19     50        1    3    1        5 0.2974405
% 20    500        1    3    1        5 0.2974405
% 22      5       10    3    1        5 0.2974405
% 23     50       10    3    1        5 0.2974405
% 24    500       10    3    1        5 0.2974405
% 27     50        1    4    1        5 0.2974405
% 28    500        1    4    1        5 0.2974405
% 31     50       10    4    1        5 0.2974405
% 32    500       10    4    1        5 0.2974405
% 35     50        1    5    1        5 0.2974405
% 36    500        1    5    1        5 0.2974405
% 39     50       10    5    1        5 0.2974405
% 40    500       10    5    1        5 0.2974405
% 163    50        1    1    1       10 0.2974405
% 164   500        1    1    1       10 0.2974405
% 167    50       10    1    1       10 0.2974405
% 168   500       10    1    1       10 0.2974405
% 171    50        1    2    1       10 0.2974405
% 172   500        1    2    1       10 0.2974405
% 175    50       10    2    1       10 0.2974405
% 176   500       10    2    1       10 0.2974405
% 179    50        1    3    1       10 0.2974405
% 180   500        1    3    1       10 0.2974405
% 183    50       10    3    1       10 0.2974405
% 184   500       10    3    1       10 0.2974405
% 187    50        1    4    1       10 0.2974405
% 188   500        1    4    1       10 0.2974405
% 191    50       10    4    1       10 0.2974405
% 192   500       10    4    1       10 0.2974405
% 195    50        1    5    1       10 0.2974405
% 196   500        1    5    1       10 0.2974405
% 199    50       10    5    1       10 0.2974405
% 200   500       10    5    1       10 0.2974405
% 237     1       10    5    6       10 0.2974443
% 10      5        1    2    1        5 0.2977492
% 14      5       10    2    1        5 0.2977492
% 34      5        1    5    1        5 0.2977605
% 38      5       10    5    1        5 0.2977605
% 174     5       10    2    1       10 0.2980465
% 18      5        1    3    1        5 0.2980540
% 342     5       10    3    1       30 0.2983439
% 30      5       10    4    1        5 0.2983627
% 241     1        1    1   10       10 0.2986488
% 178     5        1    3    1       10 0.2989724
% 221     1       10    3    6       10 0.2995634
% 229     1       10    4    6       10 0.2998645
% 338     5        1    3    1       30 0.3005345
% 245     1       10    1   10       10 0.3010765
% 401     1        1    1   10       30 0.3014152
% 13      1       10    2    1        5 0.3017389
% 29      1       10    4    1        5 0.3017502
% 269     1       10    4   10       10 0.3020137
% 9       1        1    2    1        5 0.3020476
% 217     1        1    3    6       10 0.3020551
% 449     1        1    2   31       30 0.3022998
% 265     1        1    4   10       10 0.3023449
% 89      1        1    2   10        5 0.3026423
% 93      1       10    2   10        5 0.3026423
% 85      1       10    1   10        5 0.3026498
% 81      1        1    1   10        5 0.3029547
% 253     1       10    2   10       10 0.3031843
% 417     1        1    3   10       30 0.3035757
% 1       1        1    1    1        5 0.3035945
% 25      1        1    4    1        5 0.3039107
% 173     1       10    2    1       10 0.3045092
% 385     1        1    4    6       30 0.3050512
% 461     1       10    3   31       30 0.3051076
% 5       1       10    1    1        5 0.3054464
% 249     1        1    2   10       10 0.3059621
% 169     1        1    2    1       10 0.3060035
% 389     1       10    4    6       30 0.3066659
% 49      1        1    2    6        5 0.3069557
% 53      1       10    2    6        5 0.3069557
% 165     1       10    1    1       10 0.3082167
% 33      1        1    5    1        5 0.3085328
% 37      1       10    5    1        5 0.3085328
% 161     1        1    1    1       10 0.3091388
% 193     1        1    5    1       10 0.3091652
% 393     1        1    5    6       30 0.3103245
% 197     1       10    5    1       10 0.3119429
% 189     1       10    4    1       10 0.3121876
% 381     1       10    3    6       30 0.3128162
% 185     1        1    4    1       10 0.3134410
% 353     1        1    5    1       30 0.3134937
% 469     1       10    4   31       30 0.3140432
% 21      1       10    3    1        5 0.3140696
% 377     1        1    3    6       30 0.3140808
% 429     1       10    4   10       30 0.3143255
% 17      1        1    3    1        5 0.3143744
% 397     1       10    5    6       30 0.3149466
% 421     1       10    3   10       30 0.3155751
% 453     1       10    2   31       30 0.3164521
% 209     1        1    2    6       10 0.3167871
% 409     1        1    2   10       30 0.3167871
% 445     1       10    1   31       30 0.3171183
% 345     1        1    4    1       30 0.3174533
% 413     1       10    2   10       30 0.3182739
% 333     1       10    2    1       30 0.3183416
% 213     1       10    2    6       10 0.3186352
% 321     1        1    1    1       30 0.3192675
% 329     1        1    2    1       30 0.3193127
% 405     1       10    1   10       30 0.3195310
% 201     1        1    1    6       10 0.3195912

\begin{verbatim}
361     1        1    1    6       30 0.3199074
477     1       10    5   31       30 0.3201558
177     1        1    3    1       10 0.3205360
369     1        1    2    6       30 0.3239235
45      1       10    1    6        5 0.3245220
357     1       10    5    1       30 0.3245897
41      1        1    1    6        5 0.3248269
205     1       10    1    6       10 0.3251167
181     1       10    3    1       10 0.3251618
349     1       10    4    1       30 0.3316057
365     1       10    1    6       30 0.3316095
325     1       10    1    1       30 0.3322418
337     1        1    3    1       30 0.3353395
373     1       10    2    6       30 0.3367773
341     1       10    3    1       30 0.3408913
\end{verbatim}





\end{document}
