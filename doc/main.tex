\documentclass[a4paper]{article}

\usepackage[a4paper,  margin=1.0in]{geometry}

\usepackage{graphicx}
\usepackage{float}
\usepackage{hyperref}


\usepackage[utf8]{inputenc}
\begin{document}


\title{Prediction of student's alcohol consumption with random forests}

\author{Mikołaj Ciesielski, Michał Sypetkowski}
\maketitle


\section{Data}

Research is done with dataset: \url{https://www.kaggle.com/uciml/student-alcohol-consumption/}.

The data were obtained in a survey of students
math and portuguese language courses in secondary school.
It contains a lot of interesting social,
gender and study information about students.

There are several (382) students that belong to both datasets.
These students can be identified by searching for identical attributes
that characterize each student, as shown in the annexed R file.

The dataset contains in total 395 math-course-students samples and 
649 portugese-language-studens samples.


\section{Clustering Dalc and Walc attributes into one binary attribute}

We aim to build a model that would perform binary classification --
whether student can be considered drinking alcohol regularly or not.
The dataset provides 2 attributes:
\begin{itemize}
    \item Dalc - workday alcohol consumption (numeric: from 1 - very low to 5 - very high)
    \item Walc - weekend alcohol consumption (numeric: from 1 - very low to 5 - very high)
\end{itemize}
2D histogram visualization is shown on figure \ref{fig:hist2D}.

% TODO: think about following
% We maximize consistency within 2 clusters of data using Silhouette coeficient
% (\url{https://en.wikipedia.org/wiki/Silhouette_(clustering)}).
% We use L2 distance.

\begin{figure}[h]
    \caption[]{2D histogram of Dalc and Walc attributes}
    \centering
    \includegraphics[page=1,width=1.0\textwidth]{../Rplots.pdf}
    \label{fig:hist2D}
\end{figure}


\end{document}
